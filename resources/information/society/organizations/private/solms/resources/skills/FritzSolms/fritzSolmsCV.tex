
\documentclass[10pt]{article}

\usepackage{calc}

\makeatletter
\newlength{\bibhang}
\setlength{\bibhang}{1em}
\newlength{\bibsep}
 {\@listi \global\bibsep\itemsep \global\advance\bibsep by\parsep}
\newenvironment{bibsection}
    {\minipage[t]{\linewidth}\list{}{%
        \setlength{\leftmargin}{\bibhang}%
        \setlength{\itemindent}{-\leftmargin}%
        \setlength{\itemsep}{\bibsep}%
        \setlength{\parsep}{\z@}%
        }}
    {\endlist\endminipage}
\makeatother

% Layout: Puts the section titles on left side of page
\reversemarginpar

%
%         PAPER SIZE, PAGE NUMBER, AND DOCUMENT LAYOUT NOTES:
%
% The next \usepackage line changes the layout for CV style section
% headings as marginal notes. It also sets up the paper size.

\usepackage[paper=a4paper,
            includefoot, % Uncomment to put page number above margin
            marginparwidth=30.5mm,    % Length of section titles
            marginparsep=1.5mm,       % Space between titles and text
            margin=25mm,              % 25mm margins
            includemp]{geometry}

%% More layout: Get rid of indenting throughout entire document
\setlength{\parindent}{0in}

%% This gives us fun enumeration environments. compactitem will be nice.
\usepackage{paralist}

%% Reference the last page in the page number
%
% NOTE: comment the +LP line and uncomment the -LP line to have page
%       numbers without the ``of ##'' last page reference)
%
% NOTE: uncomment the \pagestyle{empty} line to get rid of all page
%       numbers (make sure includefoot is commented out above)
%
\usepackage{fancyhdr,lastpage}
\pagestyle{fancy}
%\pagestyle{empty}      % Uncomment this to get rid of page numbers
\fancyhf{}\renewcommand{\headrulewidth}{0pt}
\fancyfootoffset{\marginparsep+\marginparwidth}
\newlength{\footpageshift}
\setlength{\footpageshift}
          {0.5\textwidth+0.5\marginparsep+0.5\marginparwidth-2in}
\lfoot{\hspace{\footpageshift}%
       \parbox{4in}{\, \hfill %
                    \arabic{page} of \protect\pageref*{LastPage} % +LP
%                    \arabic{page}                               % -LP
                    \hfill \,}}

% Finally, give us PDF bookmarks
\usepackage{color,hyperref}
\definecolor{darkblue}{rgb}{0.0,0.0,0.3}
\hypersetup{colorlinks,breaklinks,
            linkcolor=darkblue,urlcolor=darkblue,
            anchorcolor=darkblue,citecolor=darkblue}

%%%%%%%%%%%%%%%%%%%%%%%% End Document Setup %%%%%%%%%%%%%%%%%%%%%%%%%%%%


%%%%%%%%%%%%%%%%%%%%%%%%%%% Helper Commands %%%%%%%%%%%%%%%%%%%%%%%%%%%%

% The title (name) with a horizontal rule under it
%
% Usage: \makeheading{name}
%
% Place at top of document. It should be the first thing.
\newcommand{\makeheading}[1]%
        {\hspace*{-\marginparsep minus \marginparwidth}%
         \begin{minipage}[t]{\textwidth+\marginparwidth+\marginparsep}%
                {\large \bfseries #1}\\[-0.15\baselineskip]%
                 \rule{\columnwidth}{1pt}%
         \end{minipage}}

% The section headings
%
% Usage: \section{section name}
%
% Follow this section IMMEDIATELY with the first line of the section
% text. Do not put whitespace in between. That is, do this:
%
%       \section{My Information}
%       Here is my information.
%
% and NOT this:
%
%       \section{My Information}
%
%       Here is my information.
%
% Otherwise the top of the section header will not line up with the top
% of the section. Of course, using a single comment character (%) on
% empty lines allows for the function of the first example with the
% readability of the second example.
\renewcommand{\section}[2]%
        {\pagebreak[2]\vspace{1.3\baselineskip}%
         \phantomsection\addcontentsline{toc}{section}{#1}%
         \hspace{0in}%
         \marginpar{
         \raggedright \scshape #1}#2}

% An itemize-style list with lots of space between items
\newenvironment{outerlist}[1][\enskip\textbullet]%
        {\begin{itemize}[#1]}{\end{itemize}%
         \vspace{-.6\baselineskip}}

% An environment IDENTICAL to outerlist that has better pre-list spacing
% when used as the first thing in a \section
\newenvironment{lonelist}[1][\enskip\textbullet]%
        {\vspace{-\baselineskip}\begin{list}{#1}{%
        \setlength{\partopsep}{0pt}%
        \setlength{\topsep}{0pt}}}
        {\end{list}\vspace{-.6\baselineskip}}

% An itemize-style list with little space between items
\newenvironment{innerlist}[1][\enskip\textbullet]%
        {\begin{compactitem}[#1]}{\end{compactitem}}

% To add some paragraph space between lines.
% This also tells LaTeX to preferably break a page on one of these gaps
% if there is a needed pagebreak nearby.
\newcommand{\blankline}{\quad\pagebreak[2]}

% 

%%%%%%%%%%%%%%%%%%%%%%%% End Helper Commands %%%%%%%%%%%%%%%%%%%%%%%%%%%

%%%%%%%%%%%%%%%%%%%%%%%%% Begin CV Document %%%%%%%%%%%%%%%%%%%%%%%%%%%%

\begin{document}
\makeheading{Curriculum Vitae: Dr Friedemann (Fritz) Solms}

\section{Contact Information}
%
% NOTE: Mind where the & separators and \\ breaks are in the following
%       table.
%
% ALSO: \rcollength is the width of the right column of the table
%       (adjust it to your liking; default is 1.85in).
%
\newlength{\rcollength}\setlength{\rcollength}{1.85in}%
%
\begin{tabular}[t]{@{}p{\textwidth-\rcollength}p{\rcollength}}
33 Jenvey Rd      & \textit{Cell:} 072 128 2314 \\
Summerstrand            & \textit{Tel:} 041 583 4832 \\
Port Elizabeth           & \textit{E-mail:} \href{mailto:fritz@solms.co.za}{fritz@solms.co.za}\\
6001    & \textit{WWW:} \href{http://www.solms.co.za/}{www.solms.co.za}\\
\end{tabular}

\section{Objective}
%
Acceptance for PhD in Computer Science 

\section{Citizenship}
%
South African and German (RSA ID No: 631017 5157 087)

\section{Family}
%
Married to Ellen Lawton Solms with two children, Edwin and Alexander Solms.

\section{Research Interests}
%
Model-Driven Development, Architecture, Technology Neutral Design (URDAD)

\blankline

\section{Degrees}
%
PhD. Theoretical Physics, University of Pretoria
        \begin{innerlist}
	  \item 1988-1991
	  \item Thesis Topic: \emph{Finite Size effects in Poly-Crystalline high-temperature superconductors}
	  \item Supervisor: Professor Hank Miller
        \end{innerlist}

\blankline

MSc Theoretical Physics, UNISA
        \begin{innerlist}
	  \item Thesis Topic: \emph{An Improved SCF Iteration Scheme} (67\%)
	  \item Supervisor: Dr Piotr Badziag
	  \item Courses: Advanced Solid State Physics (I \& II), Advanced Quantum Mechanics, Particle Physics, 
Quantum Field Theory (Avg: 69\%)
        \end{innerlist}

\blankline

BSc Honours, University of Pretoria
        \begin{innerlist}
	  \item 1985
	  \item Physics
        \end{innerlist}

\blankline

BSc, University of Pretoria
        \begin{innerlist}
	  \item 1982-1984
	  \item primary subjects (taken to 3rd year level): Mathematics and Physics
	  \item secondary subjects: Applied Mathematics and Chemistry
        \end{innerlist}

\section{Additional courses}
Courses passed for non-degree purposes:
\begin{innerlist}
  \item OMG Advanced UML Certification, 2005
  \item Psychology 1a and 2a (Rand Afrikaans University, 1994)
  \item Theoretical Computer Science I \& II (UNISA, 1987/8)
  \item Computer Programming I \& II (UNISA, 1987/8)
  \item Informatics I (1987)
  \item Syferrekenaar 330 (University of Pretoria, 1984)
\end{innerlist}

\section{Bursaries}
Bursaries received:
\begin{innerlist}
  \item Foundation of Research and Development, 1990-1992
  \item Pretoria Portlans Cement, 1982-1984
\end{innerlist}

\section{Journal Papers}
Publications in refereed scientific journals:
    \begin{innerlist}
      \item  Fritz Solms and Dawid Loubser, URDAD as a semi-formal approach to analysis and design,
	     Journal Innovations in Systems and Software Engineering, Springer London, 1614-5046, 
	     1 January 2010.
      \item Fritz Solms and Dawid Loubser, Generating MDA's platform independent model using URDAD,
	    Knowledge-Based Systems, vol 22, 174--185, 2009.
      \item F. Solms and W.-H. Steeb, Distributed Monte Carlo Integration using CORBA and Java,
	    International Journal of Modern Physics C, vol. 9, 903--915, 1998.
		\item F. Solms, P.G.W. van Rooyen and J.S. Kunicki
      \item W.-H. Steeb, F. Solms, Tan Kiat Shi and R. Stoop, Cubic Map, Complexity and Ljapunov Exponent,
	    Physica Scripta, vol. 55, 520--522, 1997.
      \item W.-H. Steeb and F. Solms, Complexity, Chaos and One-Dimensional Maps,
	    South African Journal of Science, vol. 92, 353--354, 1996
		\item F. Solms, P.G.W. van Rooyen and J.S. Kunicki, Maximum Entropy Performance Analysis of Spread Spectrum
				Multiple-Access Communications, in Maximum Entropy and Bayesian Methods, pp101--108, Kluwer, Academic Publishers, 1996.
      \item F. Solms, P.G.W. van Rooyen and J.S. Kunicki, Maximum Entropy and Minimum Relative Entropy in 
	    Performance Evaluation of Digital Communication Systems, IEE Proceedings: Communications, 
	    vol. 142, no. 5, 250--254, August 1995.
      \item W.-H. Steeb W-H, F. Solms and K. S. Tan, Genetic Algorithms and Object-Oriented Programming,
	    Int. J. Mod. Phys. C, 853--869, vol. 6, 1995.
      \item F. Solms, R.M. Quick and H.G. Miller, Finite Size Effe cts and Polycrystalline high-Tc Materials,
	    Physical Review B, Vol 49, 15945-15951, 1994.
      \item W.-H. Steeb, F. Solms and R. Stoop, Chaotic Systems and Maximum Entropy Formalism,
	    Journal of Physics A, L399--L402, vol. 27, 1994.
      \item M. Marinus, H.G. Miller, R.M. Quick, F. Solms and D.M. van der Walt, Order Parameter for pairing systems, 
	    Physical Review C, vol. 48, 1713--1718, 1993.
      \item F. Solms, N.J. Davidson, H.G. Miller, R.M. Quick and H.L. Gaigher, BCS and Polycrystalline high-Tc materials,
	    Physics Letters A, vol 170, 84--88, 1992.
		\item R. Rossignoli, R.M. Quick, H.G. Miller and F.Solms, Correlated finite temperature BCS approximation in
			 finite systems, Physics Letters A, vol 167, 84--88, 1992.
      \item S.M. Peres, R.M. Quick, N.J. Davidson, H.G. Miller and F. Solms, Reconstruction of the Spin Dependence of 
	    One Neutron Transfer Spectroscopic Sums from Incomplete Information, Physical Review C, Vol 45, 870-872, 1992.
      \item E.D. Malaza, R.A. Ritchie, F. Solms, D.W. von Oertzen and H.G. Miller, Predictions of LEP hadronic
	    multiplicity distributions using the maximum entropy principle, Physics Letters B, vol 266, 169--, 1991.
      \item F. Solms, H.G. Miller and A. Plastino, Geometric Treatment of Finite Size Effects in Interacting Systems, 
	    Physics Letters A, vol 157, 286-289, 1991.
      \item P. Badziag and F. Solms, An Improved SCF Iteration Scheme, Computers and Chemistry, vol 12, 233--236, 1988.
    \end{innerlist}

\section{Conferences}
Publications in refereed and non-refereed conference proceedings:
    \begin{innerlist}
      \item Fritz Solms and Dawid Loubser, URDAD as a Semi-Formal Approach to Analysis and Design, 
	    presented at the International Conference on Formal Engineering Methods, ICFEM'09, Rio de Janeiro,December 2009.
      \item Fritz Solms, Technology Neutral Business Process Design using URDAD,
	    New Trends in Software Methodologies, Tools and Techniques: Frontiers in Artificial Intelligence and Applications 
	    IOS Press, vol 161, 52--70, 2007 .
      \item F. Solms, E. Smit and Z.J. Nel, A neural network diagnostic tool for the chronic fatigue syndrome,
	    IEEE International Conference on Neural Networks, Washington DC,
	    vol. 2, 778--781, 1996.
      \item P. van Rooyen and F. Solms, Maximum entropy investigation of the inter user interference distribution in a DS/SSMA system,
	    Personal, Indoor and Mobile Radio Communications, 
	    Sixth IEEE International Symposium on Wireless: Merging onto the Information Superhighway,
	    1308--, 1995.
      \item F. SOLMS, P.G.W. van Rooyen, J.S. Kunicki, Maximum entropy and performance analysis of spread spectrum
	    multiple access systems, MaxEnt'94, Proceedings of the 14th International Maximum Entropy Workshop,
            St John's College, Cambridge, England, August 1994.
      \item F. Solms, R.M. Quick and H.G. Miller, Finite Size Effects in Polycrystalline high-Tc materials, 
	    Proceedings of the XVIII International Workshop on Condensed Matter Theories, Valencia, Spain, 6-10 June 1994.
      \item F. Solms, P. van Rooyen and J. Kunicki, Maximum entropy and average error rates in digital communication
	    systems, COMSIG-94., Proceedings of the 1994 IEEE South African Symposium on Communications and Signal Processing,
	    11--15, Oct 1994.
      \item F. Solms, H.G. Miller and A. Plastino, Geometric Treatment of Finite Size Effects in Interacting Systems,
	    Proceedings of the 8'th International Conference on Dynamical Processes in Excited States in Solids, Leiden,
	    The Netherlands, published in Journal Of Luminescence, vol 53, 143--, 1992.
      \item F. Solms, N.J. Davidson, H.G. Miller, R.M. Quick and H.L. Gaigher, BCS and Polycrystalline high-Tc materials,
	    Proceedings of the 27'th Annual Seminar on Theoretical Physics at the SAIP meeting at Wits, Johannesburg, 1992.
      \item R.A. Ritchie, F. Solms, D.W. von Oertzen, H.G. Miller and E.D. Malaza, Predictions of LEP hadronic multiplicity
	    distributions using the maximum entropy principle, Proceedings of the 26'th Annual Seminar on Theoretical Physics
	    at the SAIP meeting in Bloemfontein, 1991.
      \item R.A. Ritchie, F. Solms D.W. von Oertzen, H.G. Miller and E.D. Malaza, The Maximum Entropy Principle Applied to
	    Hadron Multiplicity Distributions, Proceedings of the 17'th South African Symposium on Numerical Mathematics, 
	    Umhlanga Rocks, July 1991.
    \end{innerlist}

\section{Books}
Books published with international publishers:
    \begin{innerlist}
      \item Willie-Hans Steeb and Fritz Solms, Applications of C++ Programming,
	    World Scientific Press, Singapore, 1995, ISBN 981-02-2313-7 - reprinted in 2000.
    \end{innerlist}

\section{Patents}
Co-owner of the following patent:
    \begin{innerlist}
      \item SA Patent Application No. 95/3147, "The Determination of Individual Gas Concentrations", Friedemann (Fritz) Solms
      and Campbell John Cairns, filed 19 April 1995
    \end{innerlist}
\blankline

\section{Invited Speaker}
Selected invites as guest speaker at conferences and meetings:
    \begin{innerlist}
      \item Enterprise Architecture in the Context of a Services Oriented Enterprise, 
	    Enterprise Architecture Practitioners Conference, Gallagher Estates, 1-2 September 2009.
      \item Enterprise Architecture, Meeting of the International Institute of Business Analysts, JSE, 16 September 2008.
    \end{innerlist}

\section{Courses Developed \& Presented}

  Applied Mathematics courses
  \begin{innerlist}
    \item Information Theory and Maximum Entropy Inference
    \item Neural Networks and Fuzzy Logic
    \item Numerical methods with applications in C++ (second year)
  \end{innerlist}

  Architecture and Design courses
  \begin{innerlist}
    \item Information Theory and Maximum Entropy Inference
    \item Neural Networks and Fuzzy Logic
    \item Numerical methods with applications in C++ (second year)
    \item Architecture
    \item Enterprise Architecture
    \item Object-Oriented Analysis and Design using UML and URDAD
    \item Design Patterns
    \item Enterprise Integration
  \end{innerlist}

\blankline

  Business Analysis courses
  \begin{innerlist}
    \item Business Analysis using UML and URDAD
    \item Organization Architecture
  \end{innerlist}

\blankline

  Technology courses
  \begin{innerlist}
    \item Java, Advanced Java, Enterprise Java Beans, Java EE Presentation Layer Development
    \item XML and Web Services
    \item Object-Oriented Programming using C++, Object-Oriented Programming using ANSI-C
    \item Linux
  \end{innerlist}

\blankline

  Scientific courses
  \begin{innerlist}
    \item Numerical Methods (2nd year level)
    \item Maximum Entropy Inference (4th year level)
    \item Neural networks (4th year level)
  \end{innerlist}

\section{Employment}
Solms Training \& Consulting (2000--present)
\begin{innerlist}
   \item Founder and CEO of company
   \item Focus on organizational and systems architecture, technology neutral design and open standards based IT technologies.
   \item Training and consulting services for the corporate sector.
   \item Strong emphasis on research.
\end{innerlist}

\blankline

IBM South Africa (1999, part-time)
\begin{innerlist}
  \item Trainer in Object-Oriented Technologies (presenting OOAD, Java and C++ courses)
\end{innerlist}

\blankline

Standard Corporate and Merchant Bank (1998--2000)
\begin{innerlist}
  \item Mathematical modeling of financial instruments, their pricing and risk profiling.
  \item Software architect, lead designer and developer for the Quantitative Analysis team.
  \item Pricing and risk analysis with particular focus on the derivatives market.
  \item Architect of core of trading system which was later used by Standard Bank, Brazil.
  \item Liason with business and clients.
\end{innerlist}

\blankline

Rand Afrikaans University (1992--1998)
\begin{innerlist}
  \item Senior lecturer in Applied Mathematics
  \item Co-founder of the International School for Scientific Development
\end{innerlist}

\blankline

South African Microelectronic Systems (SAMES) (1988)
\begin{innerlist}
  \item Quality assurance specialist with some limited software development responsibilities.
\end{innerlist}

\blankline

UNISA (1986/7)
\begin{innerlist}
  \item Research assistent and lead developer of the Solid State Physics research group.
\end{innerlist}


\section{References}
%
\begin{tabular}{ll}
  Prof C.M. Viller & Prof W.H. Steeb  \\
  Applied Mathematics & Applied Mathematics \\
  University of Johannesburg & University of Johannesburg \\
  cmvillet@uj.ac.za & whsteeb@uj.ac.za \\
  \\
  Prof H.G. Miller  &  Dr. M. van Rooyen \\
 Theoretical Physics & Head of Quantitative Analysis \\
 University of Pretoria & Standard Bank \\
  hmiller@maple.up.ac.za & marchand.vanrooyen@gmail.com \\
\end{tabular}

\end{document}

%%%%%%%%%%%%%%%%%%%%%%%%%% End CV Document %%%%%%%%%%%%%%%%%%%%%%%%%%%%%
