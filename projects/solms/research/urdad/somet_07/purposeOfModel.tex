\section{The purpose of the model}

The proposed model is a model for a simple library system. Since URDAD is a services-oriented analysis and
design methodology which is symmetrical across levels of granularity, it is sufficient to look at a single service
at a particular level of granularity. The model contains a services contract and process specification for these
\verb+library::memberManagement::MemberManagement::registerMember+ service/use case together with
the minimal required elements of the lower level services from which the \verb+registerMember+ use case
is assembled. 

This use case is representative of a typical use case of a simple enterprise system as it requires
\begin{itemize}
  \item persistence and retrieval of business entities,
  \item the existence of a pre-condition and hence the potential raising of an exception,
  \item pre and post conditions which rely on other services,
  \item the presence of UML specialization, composition and association relationships on entities,
\end{itemize}

The purpose of the model is to provide a simple, yet realistic model for a simple enterprise system. 

\begin{description}
  \item[Simple, yet realistic] The model is very simple as not to introduce unnecessary complexities
		for the model transformation without being overly simple such that it does not contain the typical complexities
		of enterprise systems.

  \item[Complete and consistent] The model should be a complete functional requirements and process specification
in the sense that given a target architecture specification, the model should facilitate a complete implementation mapping.
\note{The OCL encoding of the pre-and post-conditions of the service has been done, but has not yet been tested. 
The OCL invariance constraints to specify how the request objects for the lower level services are to be constructed
in the higher level service still needs to be done. These will be provided if our model case is selected.}

  \item[Technology neutral] The model is meant to be technology neutral and should facilitate implementation
				mappings across technologies and architectures. The model transformations should facilitate full 
				implementation mappings onto commonly used enterprise architectures like Java-EE and SOA.

  \item[Neutral of access channels] The model has the services and supporting business processes and
	 value object definitions, but does not specify any particular access channels. The architectural requirements
	 should contain the integration requirements which contains the specification of the required access channels.
	 The architecture specification must then contain the specification of the adapter technologies for these required access 
	 channels (e.g.\ application, web and mobile device based presentation layers for human access, CORBA and web services access channels
	 for system access, ...).

  \item[Support model validation] The model should be such that it facilitates model validation, testing for consistency,
		completeness and correct usage of URDAD methodology.

  \item[Documentation generation] The model should have sufficient information and a well defined model structure
	 and semantics to support full natural language documentation generation and full UML to English mapping.
\end{description}
