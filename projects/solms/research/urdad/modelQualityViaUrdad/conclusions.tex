\section{Conclusions and future work}
\label{sec:conclusions}

The URDAD methodology enforces some of the accepted design principles which promote model quality. The process provides a simple design algorithm with specified inputs and outputs for each design step. The resultant PIM which has a specified model structure addresses many of the stake holder quality requirements for the PIM. The PIM  confines complexity, provides bi-directional traceability and enforces contracts based decoupling,  testability and reusability. The defined model structure facilitates testing for completeness and consistency and simplifies model utility like model transformation in order to generate implementation, tests or documention.

Future work includes the completion of the model validation suites, upgrading of documentation generation tools, the development of URDAD centric tools which can improve flexibility and can enforce the URDAD model structure and various transformation tools which can be used to generate platform specific models and code.