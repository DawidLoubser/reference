\section{PIM stake holders and their quality requirements}
\label{sec:qualityRequirements}

We take the approach that there is no absolute quality, but that quality is a measure of the extend to which the stake holder's quality
requirements are fulfilled. Thus, before we can assess the quality of a (business) model, we need to identify the stake holders who
have an interest in the model and then we need to elicit their quality requirements for the model.

The stake holders who have an interest in the PIM include:
\begin{itemize}
\item \textit{Client/Business} which gets the return on investment from the product.
\item \textit{(Business) Analysts} who are responsible for performing the stake holder requirements analysis and the technology neutral (business) process design itself.
\item \textit{Architecture} which is responsible for designing a suitable infrastructure hosting the functionality (business processes) in such a way that it enables the organization/system to realize its vision and mission. 
\item \textit{Implementation} which is responsible for performing the model transformations and implementation mapping. (In a business this includes developers who develop the automated implementation mapping as well as managers who train their staff to perform certain business process steps manually).
\item \textit{Quality assurers} who are responsible for assessing the extend to which the model realizes the stake holder's functional and non-functional requirements and reporting of any defects.
\item \textit{Operations} which is responsible for executing / overseeing the execution of the (business) processes and ensuring that the services are rendered to the user's satisfaction.
\end{itemize}

Table \ref{tab:requirements} shows the PIM qualities required by these stake holders.
The results are based on an abstraction of an empirical study which included interviews with the various stakeholders in the development process.

\begin{table*}[htb]
  \caption{Stake holders and their quality requirements. \label{tab:requirements}}
	{\small
  \begin{tabular}{|l||c|c|c|c|c|c|}
    \hline
	{\bf quality} & Client/Business & Business Analysis & Architecture & Implementation & Quality assurance & Operations \\ \hline
	completeness  &   & x & x & x & x & x \\ \hline
	consistency   &   & x & x & x & x & x \\ \hline
	simplicity/understandability     &   & x &   & x & x & x \\ \hline
	modifiability & x & x &   &   &   &   \\ \hline
	reusability   &   & x &   & x &   &   \\ \hline
	testability   &   & x &   & x & x &   \\ \hline
	traceability  &   & x &   & x &   &   \\ \hline
	cohesion      &   & x &   & x &   &   \\ \hline
  \end{tabular}}
\end{table*}