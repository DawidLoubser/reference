
\section{Introduction}
\label{sec:Introduction}

\emph{Model Driven} $\{$Architecture/Engineering/Development$\}$ (MDA, MDE, MDD)
\cite{schmidt:modelDrivenEngineering, braek:modelDrivenServiceEngineering, frankel_2003:enterpriseMDA},
 as an academic idea has has not seen a practical translation into the IT and software engineering industry to the extent which was initially envisaged. 

Part of the problem might be the inherent difficulty of the model transformation  process (i.e. refinement) on which the theory of MDD is based, which requires much deeper computer science knowledge than the average IT worker possesses. This dificulty is exacerbated by often not having a well defined model structure as input. In general the contents and structure of different UML models and it is difficult to develop useful transformation tools which can perform model refinement, implementation mapping, test and documentation generation across such widely varying input models.

Originally, tool support was envisaged as a way of circumventing this problem by equipping the semi-skilled IT worker with advanced theoretical knowledge that is embedded within ``push-button'' types of support systems. However, it soon turned out that the development of such support tools for MDSD is not a trivial task at all. Moreover, a support tool requires a flawless (i.e. syntactically correct and semantically consistent) input model before it can be further processed in an automated fashion---much like a compiler requires syntactically and semantically correct input in a high level language before it can generate target code.

Thus, we discover that even the initial, top-level models in the MDD pyramid tend to be of such poor quality that they cannot sensibly be further processed, even if the desired transformation tools were already available. In other words, the design of the initial models at the highest level of abstraction---from which, according to the theory of MDD, all other deliverables have to be generated---seems to be an intrinsically difficult and error-prone activity.

