
\section{Research Approach}
\label{sec:researchApproach}
Amongst the many papers which have already addressed this problem from various perspectives, we have
 chosen \cite{lange_2006:improvingTheQualityOfUmlModelsInPractice} and \cite{mohagheghi_2007:evaluatingQualityInModelDrivenEngineering} as the  starting points of some our work. Those papers have described quality criteria, which we will refer to as quality \emph{requirements}, that high-quality software models and software development processes should ideally fulfill (and which are typically not fulfilled in practice).

An empirical survey was done in order to determine what model quality requirements are important to which stake holders.
 
A non-empirical qualitative approach was used to determine if URDAD employes, and enforces all of the necessary design principles to generate a high quality model. 

%================================

\subsection{Building the case}
In this paper we clarify who the stake holders are of these quality requirements  of software models, as well as re-visit the actual quality requirements as defined by \cite{lange_2006:improvingTheQualityOfUmlModelsInPractice} and \cite{mohagheghi_2007:evaluatingQualityInModelDrivenEngineering}. We only use this as a starting point, and build on these quality requirements by adding additional ones. The underlying design principles that aid in the quality requirements will also be discussed.

URDAD, a quality-oriented analysis and design process \cite{solms_2009:generatingMdasPimUsingUrdad}, which was designed especially in support of MDD, will be assessed based on the quality requirements, and the underlying design principles enforced by the step by step process.

The result of this study, is to have completed an assessment of the URDAD process from a theoretical perspective, as well as a quantitative assessment of the resultent model. We argue that URDAD succesfully passed both assessment assessments, and consequently, we conjecture that the application of the URDAD process will eventually lead to software models of better quality.

%================================

\subsection{Paper Structure}
The paper is structured as follows. In section \ref{sec:contextualization} we will discuss \emph{related work} from the perspective of URDAD, in other words, \emph{similar approaches} (to software model design) aiming at \emph{similar problems} (of software model quality).

In section \ref{sec:qualityRequirements} we introduce the various stake holders who have an
interest in software models and model quality. 

Section \ref{sec:urdad} briefly recapitulates the main features and characteristics of the URDAD process as further described in \cite{solms_2009:generatingMdasPimUsingUrdad} and we then explain how the step by step URDAD process enforce the design principles mentioned.

In section \ref{sec:assessment} we compare the URDAD features against the previously mentioned quality requirements, including (\cite{lange_2006:improvingTheQualityOfUmlModelsInPractice} and \cite{mohagheghi_2007:evaluatingQualityInModelDrivenEngineering}).

Finally, in section \ref{sec:conclusions} we summarise our findings and sketch some future
work in the context of this ongoing project. 
