
\section{Context and Related Work}
\label{sec:contextualization}
There has been significant focus on the quality of software over the last few decades. Software projects were run without any formal way of approaching the software effort. As a result, software projects were notorious for failing, being late, over budget, or even delivering the wrong solution.

%================================

\subsection{Software Engineering Challenges - A Historical Perspective} Brooks \cite{brooks_1987:noSilverBullet} argued that projects largely fail because of the inherent difficulties in software. He called it the {\it"essence"} of software, which includes complexity, conformity, changeability and invisibility. These inherent properties of software undoubtedly affects the quality of the end products produced by software engineering. It is true that many advances in software engineering merely addressed the "accidents" of software, and not the "essence". As will be seen, software modeling is an attempt to solve the perceived inherent difficulties of software, and as a result, addressing various quality issues.

There were numerous initial attempts at addressing this problem. Various methodologies were created to assist the software engineering effort, giving more guidance when building software, thereby increasing the quality of the delivered solution. Avison and Fitzgerald \cite{avison_1988:informationSystemsDevelopment} cites The BCS Information Systems Analysis and Design Working Group's definition of a methodology to be a {\it"recommended collection of philosophies, phases, procedures, rules, techniques, tools, documentation, management, and training for developers of information systems"}.

All of these attempts were essentially process driven initiatives. The aim was to lay down pre-defined steps and rules to be followed, in a hope that it would increase the quality of software. Even-though formal methodologies and processes were a step in the right direction, it could not single handedly address the inherent difficulties in software, of which quality is one.

%================================

\subsection{Advancements in Software Engineering}
This was the beginning of new concepts, such as Agile Software Development, including {\it Extreme Programming (XP)} \cite{abrahamsson_2003:newDirectionsInAgileMethods}. These new concepts and philosophies in software engineering redefined the way organisations viewed and managed project resources and approaches. Another new concept that was introduced was {\it Object Oriented Analysis and Design (OOAD)}, which raised the level of abstraction at which Analysis and Design is done, thereby bridging the gap between analysis and design (RK to reference).

Another concept that arose was the idea of doing actual programming at a much higher level of abstraction, and moving closer to the knowledge domain at which the end solution is aimed. The {\it Model Driven Architecture (MDA)} \cite{frankel_2003:enterpriseMDA,siegel_2001:developingInMDA} represents the concepts of this movement, and includes the discipline of software modeling. A practitioner should be able to {\it model} a solution in some predefined notation (such as the UML), and with appropriate tool support, he should be able to do code generation instead of intensive manual programming. This approach aims at bridging the gap between analysis and design even further, by incorporating some of the design decisions in the early analysis phases already, and separating the architectural decisions from the logical design.

As this is a relatively new concept in software development, the approaches and processes aimed at producing conceptual, executable software models are still evolving. There is a need for more substantial empirical evidence of the successful, and unsuccessful implementations of these approaches in the industry. Snelting \cite{snelting_1998:pauFeyerabendUndDieSoftwareTechnologie} identified this need to test software methodologies and approaches in practice, and not to only regard it as academic advancement in software engineering. Proper empirical research will assist the evolution and maturing of new model driven approaches. There is an explicit need to be able to produce quality software models in practice, as well as being able to measure the quality in these models.

Although the measurement of software quality is important, the approaches aimed at producing quality in models are even more important from and organisational perspective. The better the quality at the early stages of the project, the more cost effective the project would be \cite{coram_2005:impactOfAgileMethodsOnProjectManagement}.

%================================

\subsection{Related Work}
There are various similiar intiatives that exist at the moment, trying to achieve the same goal, namely building high quality models in the context of the MDA. Some of them are mentioned next, in order to sketch the modeling landscape.

%--------------------------------

\subsubsection{Empirical Analysis of Architecture and Design Quality} 
Empirical Analysis of Architecture and Design Quality (EmpAnADa) \cite{lange_2004:anEmpiricalAssessmentOfCompletenessInUmlDesign} is a project that aims at developing techniques to improve the quality of models. A quality model is proposed \cite{lange_2005:managingModelQuality}, as well as a supporting prototype to manage the quality model. The model is applied in various phases of the software development effort. The quality model not only measures the quality of the model, but also that of the software system that results from the model.

%---------------------------------

\subsubsection{URDAD}
URDAD, which is aimed at addressing the many known quality issues of modeling. We believe that URDAD can address these issues as will be seen in the rest of the paper. URDAD does not explicitly measure model quality. It defines rules and steps in order to create high quality UML models in line with the Platform Independent Model (PIM) requirements of the MDA and provides a model validation suite which assesses whether the resultant UML model has the structure and content required of an URDAD PIM. It is therefore process driven, while incorporating best practice principles and rules. This will be discussed in section \ref{sec:urdad}.

