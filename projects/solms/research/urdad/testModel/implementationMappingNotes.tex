\section{Implementation mapping notes}

This section specifies some general guidelines on the envisaged model transformations outputs of an URDAD model.
The model transformation outputs include the encoding of services contracts, service implementations, unit tests
for services contracts and documentation for services.

%=======================================================================

\subsection{Implementation mapping of leaf services}

The requirements for leaf services which are sourced from outside the scope of the system are specified 
through a services contract which specifies the unputs, outputs, pre- and post-conditions and quality requirements.
These contracts need to be mapped onto technology representations of such contracts. Examples include
WSDL, CORBA IDL and Java interface encodings.

URDAD distinguishes between errors and exceptions with errors communicating that an internal problem has occured 
which prevents the service provider to meet the contractual obligations for the service, while and exception communicates
that a service provider does not provide the requested service because a pre-condition for the service was not met.
URDAD further requires that an exception class is introduced for each pre-condition.

The model contains the linkage between a pre-condition and an associated exception class in the form of a dependency.
Since UML does not specfy a notation for constraints, the dependency is generally not showable in UML diagrams. 
It is, however, specified in the model and URDAD supporting tools could show it diagrammatically. During implementation
mapping of the services contract one needs to make certain that the correct exception is specified to be raised.

%=======================================================================

\subsection{Implementation mapping of composite services}

Services are recursively assembled from lower level services until a required functionality is provided by
an externally available service, i.e.\  a service provided by the programming language, frameworks or libraries,
or external systems we are integrating with.

The service contract implementation mapping is also done for non-leaf services. In addition the process
realizing the service needs to be implemented.

The assembly of a process for a composite service from lower level services is specified in a UML composite 
activity  and visually represented in a UML activity diagram. Note that the only operations contained in an URDAD
compliant UML activity diagram are
\begin{itemize}
  \item the activity for the outer service with its input and output nodes,
  \item call operations which request lower level services,
  \item send signal operations including those for raising exceptions, and
  \item flow control activities like decision and merge nodes, forks and sysnchronization bars.
\end{itemize}

This subset is sufficient yet minimal, has well defined semantics and allows for straight forward implementation 
mappings.

%=======================================================================

\subsection{Implementation mapping of exchanged value objects}

The URDAD model contains the data structure specifications for the exchanged data objects. These
are mapped onto a technology representation like data classes in the chosen programming language, 
XML schemas if the encoding of the value objects is XML and so on.

When performing the model transformations for the implementation mapping of value objects one should
take into account the following URDAD guidelines:
\begin{description}
  \item[public attributes] In the spirit of CORBA, the URDAD model uses public attributes on entities as a short
	 hand notation for getter and setters.
  \item[composition implies encapsulation] UML composition relationships are to be implemented in such a way that
the component is encapsuled. External objects should not obtain direct access to encapsulated components. Should
they require an object (e.g.\ an account balance) they would be given a copy of the balance object, not a direct reference
or pointer to the balance object.
  \item[association] The model may contain association relationships to directly accessible objects and to remote, 
non-accessible objects. Associations to accessible objects are implemented in a standard way via references, pointers,
foreign keys etc. UML association relationships to non-accessible objects are implemented as object identifiers
 (e.g.\ URI, account number, ...).
\end{description}

%=======================================================================

\subsection{Unit test generation}
Unit tests are to be generated off service contract specifications. A unit test needs to 
 
  \begin{enumerate}
	 \item Check whether all pre-conditions are met.
	 \item Request the service and
		\begin{itemize}
		  \item if the pre-conditions are not met, check that an approapriate exception is thrown,
		  \item if the pre-conditions are met check that all post-conditions hold true after the service has been provided.
		\end{itemize}
  \end{enumerate}

%=======================================================================

\subsection{Service documentation generation}

The service documentation generation is envisaged to provide a UML to natual language mapping of an URDAD/OCL
model mapping the semantic statements in the model onto corresponding semantic statements in a natural language.
For example, if class B is specified in the model to be a specialization of class A then this is mapped onto a statement
that {\em instances of B are special instances of A} and so forth.

The envisaged complexity is not in the mapping of the individual semantic statements, but rather in the higher level
assembly of these statements into an ordered, readable, well-structured document. For this one can be guided by
the core elements of an URDAD model, i.e., that of the service contract specification, the identification of services
required to address the various pre- and post-conditions and the orchestration of those services within a defined
process.
