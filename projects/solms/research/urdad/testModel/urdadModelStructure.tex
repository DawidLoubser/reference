\section{URDAD model structure}

An URDAD model is service centric. Every service is associated with a use case and a services contract. 
Effectively URDAD treats a use case as a diagrammatic representation of a service. Services are recursively
assembled from lower level services. The leaf services are services sourced from outside the scope of the system
being modeled. These services may either be low level services provided by the programming language or employed
frameworks (e.g.\ enterprise application servers, object-relational mappers, graphic library services, ...) or services
sourced from other systems or other external service providers. In either case, the service requirements for these
leaf services are specified through a services contract.

%--------------------------------------------------------------------------------------------------------------------------

\subsection{Model elements for a service at a particular level of granularity}

URDAD fixes levels of granularity and hence exhibits the different levels of granularity directly in the model structure.
For a service/use case at any level of granularity URDAD specifies
\begin{itemize}
  \item a services contract with the inputs and outputs, pre- and post conditions and quality requirements for the service,
  \item the responsibility allocation view specifying the functions required to realize the pre and post-conditions 
specified in the services contract, and the allocation of these services to services contracts encoded as interfaces in UML.
  \item the process design showing the process details of how a service is assembled from the lower level services.
\end{itemize}

%--------------------------------------------------------------------------------------------------------------------------

\subsection{UML model encoding}

An URDAD model can be encoded in UML/OCL using
\begin{itemize}
  \item a class diagram with an interface for the services contract with the pre- and post-conditions specified through
	 OCL encoded constraints.
  \item class diagrams for exchanged entities/value objects,
  \item an activity diagram for the process specification.
\end{itemize}

In addition use case diagrams can be used to specify functional requirements and sequence diagrams to 
specify user work flows.

%--------------------------------------------------------------------------------------------------------

\subsection{Decoupling via interfaces/contracts}

URDAD enforces throughout decoupling via services contracts (encoded in UML as interfaces). In an implementation
mapping it is understood that, when plugging in a particular concrete service provider to realize the services contract,
an appropriate adapter is used (e.g.\ web services, JCA adapter, ...). If the service is to be provided by a human, a human
adapter will be used (e.g.\ an email request channel with a link to a web-based user interface).

Actors are represented in URDAD by UML interfaces and not by UML actors. This introduces one less concept for a role 
and facilitates the inclusion of contractual obligations for the external objects (including the user obligations). It also
avoids the problem of something being external (an actor) at one level of granularrity, but internal (not an actor) at another.

%--------------------------------------------------------------------------------------------------------

\subsection{Services contract specification}

A services contract is specified using a UML interface with the service, its inputs and outputs, pre- and
post-conditions and quality requirements. 

%--------------------------------------------------------------------------------------------------------

\subsection{Process specification}

In URDAD the process specification for a service is usually populated via a UML activity diagram 
which has a very restricted structure. The process specification is a control flow across lower level
services from which the higher level service is assembled. Each activity in the process specification
is thus either a call operation requesting a lower level service or a control flow activity like a decision 
or merge node, fork or synchronization bar. The inputs and outputs for the process are, of course,
the inputs and outputs for the composite service. These are shown diagrammatically as parameter
nodes.

Because the flows are control flows as executed by the control logic of the higher level service,
the input and output pins are not connected via object flows as the output of one lower level service
is not the input to the next lower level service. Instead, the output is received by the controller 
(i.e.\ the outer composite service) which then constructs the service request object for the next requested
service in the process specification from the information which is at that stage available to the process.

Each service has one initial (entry) node which starts with the receipt of the request object and two 
activity final nodes, one for normal exit and one for service abortion. 
In the case of normal exit the service is provided and resultant return value are provided. 
A service can only be aborted if one of the pre-conditions for the service is not met. In such scenarios
the exception associated with the pre-condition is raised and sent via a send signal action and
the process flow terminates on that aborted activity final node which represents service abortion.

In a complete UML/OCL encoding of an URDAD model OCL invariance constraints are used to specify how the
controller (higher level service) constructs the request object for a lower level service from the request object for
the higher level service and any information so far accumulated in the process.
