\documentclass{acm_proc_article-sp}
\usepackage{graphicx}

\begin{document}

\title{URDAD: a Quality-Driven Analysis and Design Process
       for Model-Driven Software Development}

\numberofauthors{4}

\author{
\alignauthor Fritz Solms\\
       \affaddr{Solms Training and Consulting}\\
       \affaddr{113 Barry Hertzog Ave, Emmarentia}\\
       \affaddr{2915, Johannesburg, South Africa}\\
       \email{fritz@solms.co.za}

\alignauthor Stefan Gruner\\
       \affaddr{Deptartment of Computer Science}\\
       \affaddr{University of Pretoria}\\
       \affaddr{South Africa}\\
       \email{sgruner@cs.up.ac.za}
\and
\alignauthor Dawid Loubser\\
       \affaddr{Solms Training and Consulting}\\
       \affaddr{113 Barry Hertzog Ave, Emmarentia}\\
       \affaddr{2915, Johannesburg, South Africa}\\
       \email{fritz@solms.co.za}
\alignauthor Derrick G. Kourie\\
       \affaddr{Deptartment of Computer Science}\\
       \affaddr{University of Pretoria}\\
       \affaddr{South Africa}\\
       \email{dkpourie@cs.up.ac.za}
}

\maketitle

\begin{abstract}
The amazing abstract

\end{abstract}


% A category with the (minimum) three required fields
\category{H.4}{Information Systems Applications}{Miscellaneous}
%A category including the fourth, optional field follows...
\category{D.2.8}{Software Engineering}{Metrics}[complexity measures, performance measures]

\terms{URDAD}

\keywords{Model-driven Software Development, Quality, URDAD,
model validation, formal model structure, UML.}


\section{Contextualization}

iii \cite{solms_2009:generatingMdasPimUsingUrdad}

%%\include{modelQualityRequirements}

%%\include{urdadMethodology}

\section{The URDAD Model}

Following the URDAD methodology, one constructs a URDAD model which is a well defined UML model.
This section aims to formally specify the URDAD model structure by defining a set of model constraints
which an URDAD model needs to adhere to.

An URDAD model is symmetrical across levels of granularity in that the model structure is the same for all
levels of granularity.

The central concept in an URDAD model is that of a service. Complex services are constructed by
defining a work flow across lower level services. Leaf services are specified purely via the contract
they need to realize.

In an URDAD model a use case and a service are exchangeable concepts. URDAD treats a use case as a
diagrammatic representation of a service - note that UML does not provide any other diagrammatic representation of a service.

For any service, there is the services contract and the user work flow specification. In addition there
may be the functional requirements specification.
 
For implementation mapping one only needs the services contract and the process specification. For traceability and documentation generation one requires also the <<requires>> dependencies from
stake holders or use cases to pre and post conditions as well as from pre and post conditions to functional requirements.

%===========================================================================================

\subsection{Constraints on the URDAD requirements model}

The URDAD requirements model contains the services contract specification and the user work flow specification. Below is a set of constraints which must hold true for the requirements model of
any service at any level of granularity.

The starting point for any service is the requirement for such a service or function. Such a functional
requirement is represented by a use case.

UML does not have the concept of a responsibility which is so central to design. In URDAD a responsibility
domain is represented by a services contract for that responsibility domain. High level responsibilities
will be assigned to high-level contracts which specify the requirements for high-level services whilst
lower level responsibility domains are assigned to lower level services contracts which specify lower
level services.

%-------------------------------------------------------------------------------------------

\subsubsection{Use case realized by contract service of same name }

A use case represents the informal requirement for a service. These requirements are formalized within
a services contract represented by a UML interface with a service which realizes the use case. The serviceand the use case should have the same name.



\textit{
context UseCase inv useCaseRealizedByContractServiceOfSameName:
(supplierDependency->size()>0) and
(supplierDependency->exists(d | d.oclIsKindOf(Realization) and
d.source->exists(s | s.oclIsKindOf(Operation) and
s.oclAsType(Operation).name=self.name and
s.oclAsType(Operation).interface.oclIsKindOf(Interface))))
}

%--------------------------------------------------------------------------------------------

\subsubsection{Each service receives one request parameter and returns one result object}

URDAD enforces the well known best practice of using a parameter or request object instead of
a parameter list (see, for example, \cite{fowler:refactoring}). URDAD requires further that the
request object is service specific, i.e.\ its components may be reused, but the request object itself
should not be reused so that it can be freely modified as the requirements for the service evolce.
This is enforced by requiring that the class for the request object is called
\verb+[ServiceName]Request+ whilst the class for the result object is called
\verb+[ServiceName]Result+.

\textit{
self.paramters.size()=1 AND substring(self.parameters[0].name,self.name.size()).toLower=self.name.toLower()
}



%--------------------------------------------------------------------------------------------

\subsubsection{Each pre- and post-condition is required}

%--------------------------------------------------------------------------------------------

\subsubsection{Each pre- and post- condition is specified}

%===========================================================================================

\subsection{Constraints on the URDAD design model}

%--------------------------------------------------------------------------------------------

\subsubsection{Each functional requirement of a use case is required to address one or more pre- and/or post-conditions}

%--------------------------------------------------------------------------------------------

\subsubsection{Each functional requirement is assigned to a service in a services contract represented by a UML interface}

Note: Can this be incorporated into the same requirement as above?

%===========================================================================================

\subsection{Process design}

%--------------------------------------------------------------------------------------------

\subsubsection{Each process is assigned as behaviour to a service of a class which realizes a service in a services contract}

\subsubsection{The process inputs and outputs correspond to the service inputs and outputs}

%--------------------------------------------------------------------------------------------

\subsubsection{A collaboration is assembled from call operations, send signals, decision nodes, forks and synchronization bars}

%--------------------------------------------------------------------------------------------

\subsubsection{Only unsatisfied pre-conditions lead to aborting the service}

%--------------------------------------------------------------------------------------------

\subsubsection{All functional requirements are realized within the business process}

%--------------------------------------------------------------------------------------------

\subsection{The value of all elements of each request object as well as of the return value is specified via OCL constraints as a function of the objects available to the process}


%--------------------------------------------------------------------------------------------


%%\include{openQuestions}

%%\section{Conclusions and outlook}
\label{sec:conclusions}

Over the years URDAD has strengthened its formal aspects through model validation
and increased enforcing of formal OCL based contracts. This simplifies, model testing and well as model transformation tasks like documentation generation and implementation mappings.

However, using standard UML modeling tools results in a lot of unnecessary overheads for modelers who have to construct the appropriate URDAD model structure themselves, obtaining only guidelines from the results of the model validations. The agility of URDAD can be considerably improved by
\begin{itemize}
  \item developing a URDAD front-end to UML which enforces the URDAD model structure directly and which guides modelers explicitly through the URDAD process, 
  \item extending the range of documentation generation transformations to provide suitable modelviews for different role players, and
  \item developing URDAD specific implementation mapping transformations for widely used implementation technologies in infrastructures.
\end{itemize}



\bibliographystyle{plain}  %%abbrv
\bibliography{modelSpecification}

\end{document}

