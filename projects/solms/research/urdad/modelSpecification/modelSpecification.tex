\documentclass{acm_proc_article-sp}
\usepackage{graphicx}

\begin{document}

\title{URDAD: a Quality-Driven Analysis and Design Process
       for Model-Driven Software Development}

\numberofauthors{4}

\author{
\alignauthor Fritz Solms\\
       \affaddr{Solms Training and Consulting}\\
       \affaddr{113 Barry Hertzog Ave, Emmarentia}\\
       \affaddr{2915, Johannesburg, South Africa}\\
       \email{fritz@solms.co.za}

\alignauthor Stefan Gruner\\
       \affaddr{Deptartment of Computer Science}\\
       \affaddr{University of Pretoria}\\
       \affaddr{South Africa}\\
       \email{sgruner@cs.up.ac.za}
\and
\alignauthor Dawid Loubser\\
       \affaddr{Solms Training and Consulting}\\
       \affaddr{113 Barry Hertzog Ave, Emmarentia}\\
       \affaddr{2915, Johannesburg, South Africa}\\
       \email{fritz@solms.co.za}
\alignauthor Derrick G. Kourie\\
       \affaddr{Deptartment of Computer Science}\\
       \affaddr{University of Pretoria}\\
       \affaddr{South Africa}\\
       \email{dkpourie@cs.up.ac.za}
}

\maketitle

\begin{abstract}
The amazing abstract

\end{abstract}


% A category with the (minimum) three required fields
\category{H.4}{Information Systems Applications}{Miscellaneous}
%A category including the fourth, optional field follows...
\category{D.2.8}{Software Engineering}{Metrics}[complexity measures, performance measures]

\terms{URDAD}

\keywords{Model-driven Software Development, Quality, URDAD,
model validation, formal model structure, UML.}



\section{Context and Related Work}
\label{sec:contextualization}
There has been significant focus on the quality of software over the last few decades. Software projects were run without any formal way of approaching the software effort. As a result, software projects were notorious for failing, being late, over budget, or even delivering the wrong solution.

%================================

\subsection{Software Engineering Challenges - A Historical Perspective} Brooks \cite{brooks_1987:noSilverBullet} argued that projects largely fail because of the inherent difficulties in software. He called it the {\it"essence"} of software, which includes complexity, conformity, changeability and invisibility. These inherent properties of software undoubtedly affects the quality of the end products produced by software engineering. It is true that many advances in software engineering merely addressed the "accidents" of software, and not the "essence". As will be seen, software modeling is an attempt to solve the perceived inherent difficulties of software, and as a result, addressing various quality issues.

There were numerous initial attempts at addressing this problem. Various methodologies were created to assist the software engineering effort, giving more guidance when building software, thereby increasing the quality of the delivered solution. Avison and Fitzgerald \cite{avison_1988:informationSystemsDevelopment} cites The BCS Information Systems Analysis and Design Working Group's definition of a methodology to be a {\it"recommended collection of philosophies, phases, procedures, rules, techniques, tools, documentation, management, and training for developers of information systems"}.

All of these attempts were essentially process driven initiatives. The aim was to lay down pre-defined steps and rules to be followed, in a hope that it would increase the quality of software. Even-though formal methodologies and processes were a step in the right direction, it could not single handedly address the inherent difficulties in software, of which quality is one.

%================================

\subsection{Advancements in Software Engineering}
This was the beginning of new concepts, such as Agile Software Development, including {\it Extreme Programming (XP)} \cite{abrahamsson_2003:newDirectionsInAgileMethods}. These new concepts and philosophies in software engineering redefined the way organisations viewed and managed project resources and approaches. Another new concept that was introduced was {\it Object Oriented Analysis and Design (OOAD)}, which raised the level of abstraction at which Analysis and Design is done, thereby bridging the gap between analysis and design (RK to reference).

Another concept that arose was the idea of doing actual programming at a much higher level of abstraction, and moving closer to the knowledge domain at which the end solution is aimed. The {\it Model Driven Architecture (MDA)} \cite{frankel_2003:enterpriseMDA,siegel_2001:developingInMDA} represents the concepts of this movement, and includes the discipline of software modeling. A practitioner should be able to {\it model} a solution in some predefined notation (such as the UML), and with appropriate tool support, he should be able to do code generation instead of intensive manual programming. This approach aims at bridging the gap between analysis and design even further, by incorporating some of the design decisions in the early analysis phases already, and separating the architectural decisions from the logical design.

As this is a relatively new concept in software development, the approaches and processes aimed at producing conceptual, executable software models are still evolving. There is a need for more substantial empirical evidence of the successful, and unsuccessful implementations of these approaches in the industry. Snelting \cite{snelting_1998:pauFeyerabendUndDieSoftwareTechnologie} identified this need to test software methodologies and approaches in practice, and not to only regard it as academic advancement in software engineering. Proper empirical research will assist the evolution and maturing of new model driven approaches. There is an explicit need to be able to produce quality software models in practice, as well as being able to measure the quality in these models.

Although the measurement of software quality is important, the approaches aimed at producing quality in models are even more important from and organisational perspective. The better the quality at the early stages of the project, the more cost effective the project would be \cite{coram_2005:impactOfAgileMethodsOnProjectManagement}.

%================================

\subsection{Related Work}
There are various similiar intiatives that exist at the moment, trying to achieve the same goal, namely building high quality models in the context of the MDA. Some of them are mentioned next, in order to sketch the modeling landscape.

%--------------------------------

\subsubsection{Empirical Analysis of Architecture and Design Quality} 
Empirical Analysis of Architecture and Design Quality (EmpAnADa) \cite{lange_2004:anEmpiricalAssessmentOfCompletenessInUmlDesign} is a project that aims at developing techniques to improve the quality of models. A quality model is proposed \cite{lange_2005:managingModelQuality}, as well as a supporting prototype to manage the quality model. The model is applied in various phases of the software development effort. The quality model not only measures the quality of the model, but also that of the software system that results from the model.

%---------------------------------

\subsubsection{URDAD}
URDAD, which is aimed at addressing the many known quality issues of modeling. We believe that URDAD can address these issues as will be seen in the rest of the paper. URDAD does not explicitly measure model quality. It defines rules and steps in order to create high quality UML models in line with the Platform Independent Model (PIM) requirements of the MDA and provides a model validation suite which assesses whether the resultant UML model has the structure and content required of an URDAD PIM. It is therefore process driven, while incorporating best practice principles and rules. This will be discussed in section \ref{sec:urdad}.



%%\include{modelQualityRequirements}

%%\include{urdadMethodology}

\section{The URDAD Model}

Following the URDAD methodology, one constructs a URDAD model which is a well defined UML model.
This section aims to formally specify the URDAD model structure by defining a set of model constraints
which an URDAD model needs to adhere to.

An URDAD model is symmetrical across levels of granularity in that the model structure is the same for all
levels of granularity.

The central concept in an URDAD model is that of a service. Complex services are constructed by
defining a work flow across lower level services. Leaf services are specified purely via the contract
they need to realize.

In an URDAD model a use case and a service are exchangeable concepts. URDAD treats a use case as a
diagrammatic representation of a service - note that UML does not provide any other diagrammatic representation of a service.

For any service, there is the services contract and the user work flow specification. In addition there
may be the functional requirements specification.
 
For implementation mapping one only needs the services contract and the process specification. For traceability and documentation generation one requires also the <<requires>> dependencies from
stake holders or use cases to pre and post conditions as well as from pre and post conditions to functional requirements.

%===========================================================================================

\subsection{Constraints on the URDAD requirements model}

The URDAD requirements model contains the services contract specification and the user work flow specification. Below is a set of constraints which must hold true for the requirements model of
any service at any level of granularity.

The starting point for any service is the requirement for such a service or function. Such a functional
requirement is represented by a use case.

UML does not have the concept of a responsibility which is so central to design. In URDAD a responsibility
domain is represented by a services contract for that responsibility domain. High level responsibilities
will be assigned to high-level contracts which specify the requirements for high-level services whilst
lower level responsibility domains are assigned to lower level services contracts which specify lower
level services.

%-------------------------------------------------------------------------------------------

\subsubsection{Use case realized by contract service of same name }

A use case represents the informal requirement for a service. These requirements are formalized within
a services contract represented by a UML interface with a service which realizes the use case. The serviceand the use case should have the same name.



\textit{
context UseCase inv useCaseRealizedByContractServiceOfSameName:
(supplierDependency->size()>0) and
(supplierDependency->exists(d | d.oclIsKindOf(Realization) and
d.source->exists(s | s.oclIsKindOf(Operation) and
s.oclAsType(Operation).name=self.name and
s.oclAsType(Operation).interface.oclIsKindOf(Interface))))
}

%--------------------------------------------------------------------------------------------

\subsubsection{Each service receives one request parameter and returns one result object}

URDAD enforces the well known best practice of using a parameter or request object instead of
a parameter list (see, for example, \cite{fowler:refactoring}). URDAD requires further that the
request object is service specific, i.e.\ its components may be reused, but the request object itself
should not be reused so that it can be freely modified as the requirements for the service evolce.
This is enforced by requiring that the class for the request object is called
\verb+[ServiceName]Request+ whilst the class for the result object is called
\verb+[ServiceName]Result+.

\textit{
self.paramters.size()=1 AND substring(self.parameters[0].name,self.name.size()).toLower=self.name.toLower()
}



%--------------------------------------------------------------------------------------------

\subsubsection{Each pre- and post-condition is required}

%--------------------------------------------------------------------------------------------

\subsubsection{Each pre- and post- condition is specified}

%===========================================================================================

\subsection{Constraints on the URDAD design model}

%--------------------------------------------------------------------------------------------

\subsubsection{Each functional requirement of a use case is required to address one or more pre- and/or post-conditions}

%--------------------------------------------------------------------------------------------

\subsubsection{Each functional requirement is assigned to a service in a services contract represented by a UML interface}

Note: Can this be incorporated into the same requirement as above?

%===========================================================================================

\subsection{Process design}

%--------------------------------------------------------------------------------------------

\subsubsection{Each process is assigned as behaviour to a service of a class which realizes a service in a services contract}

\subsubsection{The process inputs and outputs correspond to the service inputs and outputs}

%--------------------------------------------------------------------------------------------

\subsubsection{A collaboration is assembled from call operations, send signals, decision nodes, forks and synchronization bars}

%--------------------------------------------------------------------------------------------

\subsubsection{Only unsatisfied pre-conditions lead to aborting the service}

%--------------------------------------------------------------------------------------------

\subsubsection{All functional requirements are realized within the business process}

%--------------------------------------------------------------------------------------------

\subsection{The value of all elements of each request object as well as of the return value is specified via OCL constraints as a function of the objects available to the process}


%--------------------------------------------------------------------------------------------


%%\include{openQuestions}

%%\section{Conclusions and open questions}

URDAD aims to provide a methodology through which domain experts like business
analysts can perform the design in a technology neutral way yielding MDA's PIM.
The design is in the spirit of MDA where architecture and design are treated as
orthogonal with the technology neutral design being ultimately mapped onto a
realization within the implementation architecture and technology. The PIM
should contain sufficient information that this technology mapping can be
automated.

This paper presents some minimal requirements for the PIM. It also defines
the URDAD analysis and design methodology which generates a PIM satisfying
these requirements. In addition, the methodology has been formulated in such a
 way as to enforce sound design principles.

There are still a number of open questions around ensuring that the technology
neutral model, the PIM, does indeed have sufficient information for automated
technology mapping. In particular, one needs to standardize the stereotypes for
certain atomic services like data-capture services, persistence services,
message delivery services, and so on. Furthermore, there will be low level
algorithmic activities which need to be specified. There is quite a lot of
support and literature on how to do this, but this step is currently not
incorporated within the URDAD design methodology.


\bibliographystyle{plain}  %%abbrv
\bibliography{modelSpecification}

\end{document}

