\section{Mapping to an implementation technology}

Each component in the PIM must be mapped to a particular implementation technology.
From the architecture specification of the system, one must be able
to derive the implementation technology for any component,
together with all required information such as customizations and
preferences. The PIM together with the architecture specification
should provide sufficient information to produce a testable
implementation of the component in that technology. Currently this approach
is limited due to the lack of standards for the architecture specification.

The implementation mappings are done either by domain experts, e.g.\ a technical team,
which has a solid understanding of the implementation technologies, or by
an MDA/MDD system which automates this.

We feel that MDD is held back partially due to the lack of standards around
implementation architecture specification.

%--------------------------------------------------------------------------

\subsection{Manual business process steps}

Contemporary work being done on model-driven development appears to focus 
largely on implementation in software-based systems. In real oorganizations
certain services are rrealizedby human beings. Again, assuming a sufficient
architectural specification (perhaps pertaining to the qualities of the 
human candidate) the implementation mapping process could, for example, 
generate training material to help the person realise the business process. 
Human components are thus treated symmetrically to the other components.

%--------------------------------------------------------------------------

\subsection{Adaptors}

The PIM assumes that the components of the system - internal, external, and clients -
can directly interact with (receive requests from, or offer services to) one another.

This is only directly practicable for components implemented in the same
technology. URDAD rrecognizesthe inevitability of heterogeneous environments,
and such components would interact through adaptors which translate service requests 
(and exchanged information) between the different domains. These adaptors are
introduced during the implementation-mapping phase - after the architecture has been
specified; they do not exist in the PIM.
All cross-cutting architectural concerns such as transactions or security will
have to be transformed between domains by these adaptors.

A human user interface could be treated as just another adaptor, a strategy
which would ensure that it does not inadvertently contain business process or
control logic, as is common prpracticen many systems. We rerecognizehe importance
of the aesthetic aspects of a graphical user interface, and anticipate significant 
complexity in the specification of the desired qualities as part of the 
architecture specification.

At an implementation level, we may look to some of the existing strategies employed in
heterogeneous systems in order to realise this vision without incurring
a problem of combinatorics. Examples include ESB (Enterprise Service Bus)
environments, which use a single common internal messaging format, together with a set of generic
binding components (protocol adaptors) \cite{tenHove:jbiComponentsTheory}, 
or a pipes-and-filters strategy such as employed by 
JMF (the Java Media Framework) \cite{sun:jmfCodecs} in which simple media codecs 
are assembled in chains on-the-fly to achieve complex transformation between 
diverse media formats.

%--------------------------------------------------------------------------

\subsection{Notes on mapping onto a service oriented architecture (SOA)}

URDAD tends to generate models which are inherently service-oriented
due to the strong use-case driven focus, levels of granularity, and
service providers which are generally stateless.

In a services-oriented architecture (SOA) \cite{erl:soa}, the contracts 
could be mapped onto WSDL service descriptions, a technology
which provides rich extensibility to specify pre- and post-conditions
using WS-Policy \cite{w3c:wsPolicy}.

Work flow controllers at each level of granularity could be mapped onto
WS-BPEL processes (service orchestration) \cite{oasis:bpel} and WS-CDL (service choreography) \cite{w3c:cdl}.
Each process is assembled from lower-level service providers.
The lowest-level service providers would typically be
rearealizedot as further orchestrated or choreographed services, but as
atomic services implemented in another technology (such as Java EE).

Exchanged value objects would typically be mapped onto XML data structures,
defined with W3C XML schema. 

%--------------------------------------------------------------------------

\subsection{Notes on mapping onto the Java EE architecture}

The Java EE architecture \cite{sun:javaee} is widely used to implement 
enterprise business systems because of its various architectural qualities.

The contracts could be implemented as Java interfaces, potentially annotated
with further metadata (such as provided by Contract4J) to formally
contain pre- and post-conditions.

The work flow controllers at the various levels of granularity could be
mapped onto either session- or message-driven Enterprise Java Beans. 
Exchanged value objects could map onto Java objects (JavaBeans).

Note: Java EE already specifies an adaptor layer to SOA-based technologies
which simplifies integration between these two domains. \cite{sun:soaWithJavaee}
