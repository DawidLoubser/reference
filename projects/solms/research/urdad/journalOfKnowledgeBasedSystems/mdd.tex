\section{Model-Driven Development (MDD)}


{\em Model Driven Development} (MDD) or its more generic form, Model Driven
Engineering (MDE),
\cite{selic:pragmaticsOfModelDrivenDevelopment,schmidt:modelDrivenEngineering, france:mddUsingUml2}
is increasingly gaining the attention of both research communities and the software development industry.

In MDD one aims to capture and maintain the business processes within abstract models,
which are then mapped onto (typically software-based) implementations. The
aim is to simplify the development and maintenance of the business processes,
as well as to facilitate automation via techniques such as model execution, model transformation 
or code generation.

The {\em Object Management Group} (OMG) promotes the Model Driven Architecture
(MDA) \cite{pastor:mdaInPractice,siegel:developingInMDA,frankel:enterpriseMDA,
stahl:mdsd}
as a standard framework for MDD. It proposes the development of a
technology neutral model, the {\em Platform Independent Model} (PIM) which is
then mapped onto one's choice of implementation architecture and technologies
via standard model transformations defined for such technologies. One of the
core benefits envisaged by MDA is that the PIM will be able to survive
technology and architecture changes \cite{siegel:developingInMDA}.

\subsection{Quality of service}

Business processes need to be deployed within an environment ensuring certain qualities.
It is the responsibility of architecture to ensure that the infrastructure into which
the business processes are to be deployed will provide the required qualities
\cite{bass:softwareArchitecture}.

URDAD thus assumes the orthogonality of the technology neutral business process design and
the implementation architecture with the former addressing the functional requirements and
the latter addressing the non-functional or quality requirements.

\subsection{Model transformations}

In a model driven approach, the platform independent model is taken through one or
more model transformations which map the technology neutral business process specification
contained in the PIM onto a series of intermediate Platform Specific Models (PSMs)
and ultimately onto a concrete, deployable implementation, the Enterprise Deployment Model (EDM).
The OMG has defined for this the QVT (Queries/Views/Transformations) \cite{omg:qvt}
which is an extension of the Object Constraint Language (OCL) and contains three domain
specific transformation languages. Other languages used to specify model transformations include
the Model Transformation Language (MTL), the ATLAS Transformation Language (ATL) as well as domain-neutral
languages like the XSLT and Java.

The inputs for the implementation mappings are
\begin{enumerate}
  \item the technology neutral analysis and business process design in the form of a PIM including
			the quality requirements for the deployed services, and
  \item the Platform Description Model (PDM) specifying the implementation architecture
			and technologies including the presentation layer infrastructure (e.g.\ Struts),
			the integration infrastructure, the component/services hosts, the persistence
			infrastructure, ...
\end{enumerate}
In order to perform a model transformations one requires the understanding of the meta models of both,
the source language as well as the target language.

URDAD defines a meta model for the
structure of the PIM. The semantics required for the meta-model is defined in a UML profile for URDAD.
Part of the PIM are the service provider contracts with the quality of service requirements. These are
specified using OMG's {\em UML Profile for Modeling Quality of Service and Fault Tolerance Characteristics
and Mechanisms} \cite{omg:umlProfileQos}. Having a well-defined meta-model for the PIM significantly
simplifies the definition of model transformations required to generate the PSMs and EDM.

The weak aspect of the source domain is usually the specification for the PDM.
Due to a lack of standards around the specification of the PDM, MDA tools often support only
mappings onto certain specific reference architectures/platforms like Java EE and Microsoft.Net
with only limited control over the parameters of the defined architecture. Alternatively
or additionally they enable one to specify the PDM indirectly through a set of transformation
elements.

