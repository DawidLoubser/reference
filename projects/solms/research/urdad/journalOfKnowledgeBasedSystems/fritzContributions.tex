\documentclass{llncs}
%

%%\begin{document}
\title{URDAD as a Quality Driven Analysis and Design Methodology}
%%\subtitle{Subtitle of your contribution}

%%----Other way of representing the authors----
%%\author	{Riaan Klopper \email{(riaan.klopper2@standardbank.co.za)}\\ 
%%		Fritz Solms\email{(fritz@solms.co.za)}\\
%%		Stefan Gruner \email{(sgruner@cs.up.ac.za)}\\
%%		Derrick Kourie \email{(dkourie@cs.up.ac.za)}}
%%\institute{Department of Computer Science, \\University of Pretoria, South Africa}

\author{Riaan Klopper\inst{1}\email{(riaan.klopper2@standardbank.co.za)} \and Stefan Gruner\inst{1}\email{(sgruner@cs.up.ac.za)}\and Derrick Kourie\inst{1}{(dkourie@cs.up.ac.za)}\and Fritz Solms\inst{2}\email{(fritz@solms.co.za)}}

\institute{Department of Computer Science, University of Pretoria, South Africa,\\
%%\email{riaan.klopper2@standardbank.co.za; sgruner@cs.up.ac.za; dkourie@cs.up.ac.za} -- I moved the email addresses next to the names
\and Solms Training and Consulting, South Africa,\\
%%\email{fritz@solms.co.za}
}

\begin{document}	
\maketitle
	
\section{PIM requirements}

In order to assess whether a PIM has the required qualities, one needs to first understand
who the users of the PIM are, their use cases and then the quality requirements around these use cases. 

\begin{description}
  \item[Analysis and Design] The team performing the stake holder requirements analysis and the technology neutral design (in an organization these are typically the business analysts) require a design which is
  \begin{itemize}
 		\item understandable from the utility domain,
		\item easily modifiable without introducing excessive inconsistency risks,
		\item traceable in both directions, i.e.\ from the functional requirements through
the layers of granularity of the design realizing them and from low level design elements back to the functional requirements for which they are required and to the stake holders who require them,
		\item ability to assess the completeness of the requirements and the design,
	\end{itemize}

  \item[Quality Assurance]
		

  \item[Implementation]

  \item[Operations]


\end{description}
%========================================================================================

\section{The URDAD methodology}

URDAD \cite{Solms2007_TechnologyNeutralDesignViaUrdad, Solms2008_GeneratingPimUsingUrdad}
is an analysis and design methodology used for technology neutral design generating
the Platform Independent Model (PIM) of the Model-Driven Architecture. It is currently largely used within the business sector for technology neutral business process design.


URDAD provides a quality-driven methodology for technology neutral analysis and design. It
aims to generate a Platform Independent Model (PIM) which satisfies commonly accepted
model quality measures.
This paper provides an interpretative study assessing the quality of an URDAD generated Platform Independent Model, (PIM) and introduces quality verification techniques which can be applied to this model.

URDAD provides a quality-driven methodology for technology neutral analysis and design. It
aims to generate a Platform Independent Model (PIM) which satisfies commonly accepted
model quality measures.
This paper provides an interpretative study assessing the quality of an URDAD generated Platform Independent Model, (PIM) and introduces quality verification techniques which can be applied to this model.

URDAD, and this paper, is not intended to measure and quantify quality, but rather to define a process that could produce high quality models. URDAD is therefore a quality enforcing process for MDD, and not a quality management process. We want to show how one can entrench inherent quality attributes in MDD processes, such as URDAD. It is more than just process, it is also concepts and principles\\


The core aims of URDAD are
\begin{itemize}
  \item to make it simpler for experts from the utility domain (e.g.\ business analysts)
		  to perform the technology neutral design leading to the PIM by providing a step
		  for step design process with well defined inputs and outputs for each step (domain
		  experts need not know anything	about the implementation infrastructure,
		  i.e.\ about the architecture and technologies within which the process is to be deployed),
  \item to include quality drivers within the analysis and design methodology ensuring that
		the resultant PIM has certain desired qualities and simplifying the generation of
		quality verification tools, and
  \item to simplify the implementation mapping through a well defined structure for the PIM.
\end{itemize}

URDAD starts with the premise that architecture/infrastructure and design can be viewed as orthogonal with architecture addressing the quality requirements and design addressing the functional requirements. Ultimately the designed process needs to be deployed into some implementation architecture. If, for example, the implementation architecture

\subsection{URDAD premises}

discuss
 - request - responses
 - use case= diagrammatic representation of service
 - tie ups
 - in
cremental population of value objects
 - user obligations
 - explain call activities - tied up to services
 - explain choice of diagrams
 - PIM

%=================================================================================================



\end{document}
