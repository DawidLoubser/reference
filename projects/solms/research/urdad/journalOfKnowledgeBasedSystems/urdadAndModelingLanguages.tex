\section{URDAD and Modeling Languages}

Even though URDAD has been largely used in conjunction with the Unified Modeling Language (UML),
it is not locked into using any particular modeling language. In order for a modeling language
to be usable with the URDAD design methodology, it must have sufficient semantics to be able to
document the PIM. In particular, it requires support for
\begin{itemize}
  \item technology neutral business process modeling,
  \item services contracts,
  \item technology neutral data structure modeling,
  \item decomposition of services across levels of granularity, and
  \item declarative specification of how one data object is constructed from other data objects.
\end{itemize}

In addition the language should generate a semantically sound model which enforces consistency
across the statements made within the modeling language. Finally, it is desirable that any
modeling language used for URDAD is extensible in order to accommodate any additional semantics
required for MDD.

%----------------------------------------------
\subsection{URDAD and UML}

The Unified Modeling Language, UML, provides most modeling elements required by
URDAD. In particular, it supports
\begin{itemize}
  \item technology neutral business process design using sequence and
	activity diagrams,
  \item the specification of services contracts using UML interfaces together
	with the Object Constraint Language (OCL),
  \item technology neutral data structure specification using UML class diagrams, and
  \item the declarative specification of data object construction is done using OCL.
\end{itemize}

In addition, UML supports an underlying object model which ensures consistency
and semantic coherence across diagrams. It also has an extension mechanism,
stereotyping, which supports the introduction of refined concepts based on the
base concepts provided by UML.

One of the issues we have experienced with UML is that it does not have a convenient
notation for the concept of a service and for specifying dependencies between services.
The closest there is in UML, is the use case diagram where a use case is often interpreted
as {\em a service of value} and dependency between services is documented using
{\em include} and {\em extend} relationships. However, these use cases would still have to
be formally tied up to the services used in sequence and activity diagrams.
In URDAD services are formally linked to use cases via a realization relationship.

\subsubsection{UML adoption for technology neutral business process design}

Business analysts have been slow in adopting UML for technology
neutral business process design. This can be largely attributed to
the complexity of the UML, the perception that UML is largely useful for
modeling software systems, and the lack of simple design methodology which
provides business analysts guidance in using UML for technology neutral
business process design.

\subsection{Choice of UML diagrams}

URDAD aims to make it more feasible for business analysts to use UML by requiring
that only 4 of the 13 UML diagrams are used together with OCL constraints specification.
For the analysis phase, URDAD requires one use case diagram for the functional requirements, a sequences diagram for the user work flow, and a class diagram for the services contract.
For the design phase URDAD requires a further use case diagram for the responsibility
identification and allocation, a sequence diagram for the success scenario, an
activity diagram for the full business process specification and a class diagrams for the
collabaration context. URDAD thus mandates the use of only four of the thirteen UML diagrams.

This is a minimal, yet sufficient set of diagrams. Implementation mappings are ultimately done from the services contracts (with OCL
base pre- and post-conditions), the sequence and activity diagrams and the collaboration
context.

Communication diagrams provide an alternative view to the information contained in sequence diagrams, but are much less accessible to business analysts.
URDAD envisages the use of composite structure diagrams and component and deployment diagrams are really only for platform specific models. State charts are not used since URDAD effectively churns out a design with stateless service providers, i.e.\ a design which is largely in the spirit of services oriented approaches. Since timing diagrams are essentially a merger between
sequence diagrams and state charts, they are not used either. Interaction overview diagrams are not used as they typically traverse levels of granularity. Package and object diagrams are only
used implicitly.

%----------------------------------------------

\subsection{URDAD and BPMN}

Even though BPMN has reasonable support for business process specification
across levels of granularity (via compound activities), it does not currently
contain sufficient semantics for technology neutral business process design in
the spirit of model driven development. In particular, BPMN does not currently
have support for

\begin{itemize}
  \item solid services contracts specification, or for

  \item data structure specification (BPMN only supports artifacts whose
	structure is to be specified in some other language).
\end{itemize}

Though not required by the BPMN specification, BPMN can be modeled as a UML profile
(see, for example, the MagicDraw implementation of BPMN).
This has the benefit of being able to mix BPMN with other UML model elements.