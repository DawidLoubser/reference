\section{Semi-formal elements of URDAD}
\label{sec:semiFormalElements}

Formal methods \cite{Monin:understandingFormalMethods, hinchey:softwareEngineeringAndFormalMethods}
are precise, mathematically
rigorous methods which remove ambiguity and result in verifiable implementations.
The three core approaches are based on denotational, operational
and axiomatic semantics.

In both denotional and operational semantics, one defines the meaning of what the
program represents/does. These approaches provide the ability to formally prove
an algorithm. For complex problems the approach may, however, be very costly.
Furthermore, there could be multiple algorithms which could suitably realize the stakeholder's
requirements - yet one will lock into a particular algorithm which
can potentially be proven correct, but which may not be the best with respect to certain
quality requirements like simplicity, performance, resource-efficiency (e.g. memory
efficiency, ...) and so on. Furthermore, replacing one algorithm with another is a complex task
which typically requires a complete redefinition of the algorithm and its proofs, and
may have a wide ranging impact across the system.

In axiomatic semantics (also known as contract-based approaches) one does not provide the formal specification of a program, but instead the formal specification of the effect the program needs to have on its environment. This can be done in the form of a contracts specified via pre- and post-conditions using constraint languages like OCL
\cite{Briand:investigationOfFormalityInUmlBaedDevelopment}. In this approach, algorithms are
proven against the contract they realize, either by formally proving that a particular algorithm realizes the contract, or by generating algorithm-independent tests from the contracts
\cite{meyer:programsThatTestThemselves}. In the former approach this is usually done recursively,
by assuming at any particular level of granularity, that the lower level services used within
the algorithm do realize their respective services contracts. In the case of generating
algorithm-independent tests from the contracts, one also needs to generate a complete set of
test data which covers the entire application domain. This may be impractical or even
impossible. URDAD can be seen as an analysis and design methodology which enforces a contracts-based
approach across levels of granularity, encouraging the formal contract specification
in the form of OCL constraints.

An axiomatic/contracts based approach is more natural to model-driven development, since the
model should preferably be developed in the problem/requirements domain and not in the
technical/solution domain. Contracts are independent of particular service providers, or the algorithms they use.

However, the requirements specification in the form of formal contracts is often non-trivial. A complete contract for a high-level service may be very complex. Domain
experts (e.g. business analysts) who develop the model require a framework within which they can incrementally
refine the requirements, and the technology neutral solution (business processes). They typically do not have sufficient technical skills to formally define the services
contracts in OCL. URDAD encourages them to define the
services contracts informally (using natural language to specify constraints, for example), with OCL experts subsequently mapping such statements to OCL. 

In addition to the service contracts across levels of granularity, URDAD enforces a number
of other elements which formalize the outputs of the methodology including
\begin{itemize}
  \item an OCL validation suite which validate that the UML model complies to the
			constraints of an URDAD meta-model, and
  \item OCL validation suites for model completeness and consistency.
\end{itemize}

These validation suites enforce
\begin{itemize}
  \item the linking of pre- and post-conditions to functional/service
			requirements (use cases) through which these pre- and post-condition is
			addressed,
	\item the linking of functional requirements (use cases) to services
			in services contracts which formalize the service requirements,
	\item the linking of service process specifications to the service contracts they realize,
	\item the construction of request objects for each service
			from the information available to the (business) process at the
			stage of service request,
	\item the construction of result objects from the information available
			at the end of the process realizing the service,
	\item the specification of contracts for all role players
			including users and service providers across levels of granularity,
	\item that for each pre-condition a corresponding exception is raised,
			informing the user that the service requested is not going to be provided.
			All other scenarios satisfying the pre-conditions for the service
			lead to the provision of the result object.
\end{itemize}
The first three provide bi-directional requirements traceability. Black-box
tests can be auto-generated from the services contracts.

Note that if the levels of granularity are not carefully managed,
the specification of some of the above constraints can become very complex. The
methodology, however, aims to provide a repeatable algorithm through which the levels
of granularity are fixed in a way which minimizes complexity at any particular level
of granularity, and which maximizes re-use of services.

The validation suites enforces a complete, consistent model which has sufficient
information to perform a complete implementation mapping, provided the semantics
for the implementation architecture and technologies are provided.
