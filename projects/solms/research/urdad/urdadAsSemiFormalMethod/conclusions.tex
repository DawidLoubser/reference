\section{Conclusions and outlook}
\label{sec:conclusions}

Over the years URDAD has strengthened its formal aspects through model validation
and increased enforcing of formal OCL based contracts. This simplifies, model testing and well as model transformation tasks like documentation generation and implementation mappings.

However, using standard UML modeling tools results in a lot of unnecessary overheads for modelers who have to construct the appropriate URDAD model structure themselves, obtaining only guidelines from the results of the model validations. The agility of URDAD can be considerably improved by
\begin{itemize}
  \item developing a URDAD front-end to UML which enforces the URDAD model structure directly and which guides modelers explicitly through the URDAD process, 
  \item extending the range of documentation generation transformations to provide suitable modelviews for different role players, and
  \item developing URDAD specific implementation mapping transformations for widely used implementation technologies in infrastructures.
\end{itemize}

