\section{The URDAD methodology}
\label{sec:urdad}

URDAD
\cite{solms:generatingMdasPimUsingUrdad,
solms:urdad}
is an analysis and design methodology which has been adopted by a number
of organizations, particularly in the financial and insurance
sectors, for their technology neutral business process design  \cite{klopper:compareSoftwareMethodologies}.
It aims to provide a simple repeatable engineering process for
generating generates MDA's PIM, in the sense that the
outputs from different domain experts with similar
domain knowledge would be very similar. To achieve this,
URDAD provides an algorithmic process for analysing and
designing the technology neutral solution.

The methodology specifies a mechanism for fixing levels
of granularity, and has specified steps (modeling activities) for each
level of granularity. The type and structure of the
inputs and outputs of each step are precisely specified.

One of the core aspects of URDAD is that it requires analysis
and design to be done across levels of granularity, with both
of these done by domain experts from different specialization
areas. Thus instead of requiring a (business) analyst
to specify the entire requirements for a new service,
this responsibility is distributed across the domain experts 
touched by the service requirements.

For example, the business
process for processing an insurance claim may require services
for assessing the claim coverage and value, as well as for
settling the claim and recuperating any losses. Those are
very different responsibility domains and it is unreasonable
to expect a business analyst to be able to provide the detailed
requirements for a business process across levels of
granularity. Instead a business analyst would specify the high-level 
requirements (such as that the losses need to be recuperated)
without having to specify the detailed requirements and process
for those lower level services. He is, however, assembling the higher
level business process across those lower level services.

In particular the methodology requires that the URDAD-compliant
UML model is populated through the following views (UML diagrams):
\begin{itemize}
  \item A {\em Services contract view} as a UML class diagram containing
        the UML interface with the service signature, class diagrams
		  specifying the data structures of the request and response objects,
		  and the pre- and post-conditions and quality requiremens for the service.
	\item A {\em User workflow view} which is typically a sequence diagram showing
			the interaction (messages exchanged) between the user and the service
			provider for the use case. Both the user and service provider are
			represented by UML interfaces for the contracts which are populated
			with the services required from these role players.
	\item A {\em Functional requirements view} which shows the lower levels
			services required to address the pre- and post-conditions.
	\item A {\em Responsibility allocation view} which specifies how the services are
assigned to services contracts  via usage dependency which point from the functional requirement/use case to the services contract which hosts the corresponding service.
	\item A {\em Process view} specifying the (business) process through which the
			service is to be realized in the form of a UML activity diagram which is
			assigned to be the behavior of the service. The input and output parameter nodes correspond to the request and response objects as specified in the services contract for the service. It shows the process flow across call operations requesting the lower level services from which the (business) process is assembled.
\end{itemize}

Services are recursively constructed from lower level services
with the lowest level services being either
not domain specific (generic services like numeric addition, or infrastructure services such as persistence) or
services which are sourced externally. For these externally-sourced services, the methodology still requires the specification of a full services contract, but are otherwise
handled as black boxes.
