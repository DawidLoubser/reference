\section{What about agility?}
\label{sec:urdadAgility}

Agile approaches
\cite{martin:agileSoftwareDevelopment, agileManifesto}
like Extreme Programming \cite{beck:extremeProgrammingExplained2}
accept and even welcome continuously changing requirements in the
context of continuously changing business opportunities and
continuous growth of knowledge on how better to generate stake
holder value. They aim to be able to provide better business value
by being able to effectively operate in an environment of
continuously changing requirements.

In such a high-risk environment it is critical to manage and mitigate
risk. This is done through practices like short feedback cycles
via short releases, on-site customer, pair-programming,
enforced, automated functional (unit) testing across levels of
granularity, and continuous integration testing.

Agile methodologies typically handle the removal of any unnecessary complexity
arising from an agile approach through peer reviews and continuous refactoring
facilitated through non-ownership of all artifacts generated by the process.

%------------------------------------------------------------------

\subsection{Applying agile principles and practices in MDD}

MDD decouples the requirements and design from the implementation architecture
and technologies and increases the level of abstraction of development.
However, continuously changing requirements are core to business agility
and many of the agile practices have been ported to this higher level of abstraction
including on-site customer (with the customer communication simplified
as the design is done in the problem/business domain), pair design,
enforced and automated functional testing at the design level either via
proving the design or executing the design in model execution environments
\cite{kirshin:umlGenericModelExecutionEngine},
peer reviews and non-ownership of designs.

In addition, MDD, and MDA in particular, aim to improve agility around technology
and architecture by decoupling the design from these and automating the
implementation mapping onto the target architecture and technologies.

%------------------------------------------------------------------

\subsection{How does URDAD aim to achieve agility?}

URDAD itself is not a development process but a process for performing
the technology neutral analysis and design generating MDA's PIM. It is
typically embedded within a model-driven development process within
which some of the agile principles and practices can be absorbed. 

However, the URDAD methodology aims to increase
agility through a number of intrinsic process elements
\begin{itemize}
  \item enforced decoupling across all levels of granularity facilitated
			through enforced binding to services contracts, implied adapters
			to concrete service providers and localization of process in
			separate work flow controllers,
  \item explicit search step for service reuse resulting in lower cost and complexity reduction, 
  \item automated documentation (including UML to natural language mapping
			and UML diagram generation) from the URDAD compliant UML model, and
  \item complexity reduction through enforced responsibility localization.
\end{itemize}

