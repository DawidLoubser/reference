%\documentclass[11pt,a4paper]{book}
%\usepackage{emlines}
%\usepackage{epsf}
%\begin{document}

\chapter{Error Handling Techniques\label{chapErrorhnd}}

\index{error handling} \index{error handling!exceptions}

\section{Introduction}

It is a good idea to separate a program into distinct subsystems that
either complete successfully or fail. Thus, a limited form of local error 
checking should be implemented throughout the system in such a way
that the overheads in development time, execution time, memory 
requirements and overall complexity do not become too high. A
fault-tolerant system should be designed hierarchically, with each
\index{error handling!hierarchical}
level coping with as many errors as it can (without making the system
too contorted) and leaving the remaining errors to be handled by
higher levels of the system. 

One generally has however the problem that the error cannot be handled
locally. Note that the user of your class library knows what to do when an
error occurs, but he does not know how to detect an error in your classes.
The author of the class library knows how to detect errors, but he does not
know how the error should be handled. We thus need a global error
communication system. Typically this would be one of the following:
\begin{itemize}
   \item Error state $=$ return value of functions and object services.
   \item Classes with error state variables.
   \item Exception handling.
\end{itemize}


%=========================================================

\section{Error Communication via Return Values}

In the past it was commmon to communicate that the requested service could 
not be supplied by a return value returned by the service. Typically this 
was an integer which may have been zero if the service was supplied as 
requested and some other value otherwise. The value would supply some
information about the source of the problem.

This scenario may be ok if your direct client (i.e.\ the object from which
the service was requested) is the one who always takes over the 
responsibility of handling the situation. Often this is, however,
not the case. In many cases the error information has to be propagated 
up several layers in the calling hierarchy. 

In such situations the return-value mechanism becomes very difficult
to manage. Any of these services may generate its own code and to keep
track of where the error originated from becomes virtually impossible
when error codes are returned. 

This can be partially addressed by returning errfor reporting objects 
instead of simple error codes. These objects could contain the 
information about the source of the problem and any other relevant
information. 

But even this can become a little burdensome. The error reporting
objects have to be passed manually up the calling hierarchy and 
error checking must be done manually after each service request.

%=========================================================

\section{Error Communication via Error States}

Sometimes the information about the problem is stored within the
service provider object itself. This is typically done via
instance members which can be 
queried by the client after the return from the service. 
If an error occurs the error state variable set to a value indicating error type.

For example, the standard IOstream library of C++ has integer state variable {\bf state}.
The different bits of this state variable indicate different error states.
The error state is queried via the class methods {\bf bad()}, {\bf eof()}, {\bf fail()}.
The user can query the error state and implement his own error handling mechanism.

This mechanism does not scale well at all. Firstly, the same object may
provide services to a wide range of clients and typically it would only
keep track of the last error state. Personally I prefer even the return
value mechanism to the error state mechanism.

%=========================================================

\section{Error Communication via Exceptions}

\index{exception handling}

A more sophisticated method facilitating the split in responsibilities between error
detection and error handling is provided by C++'s support for exception handling.
Exceptions handling mechanisms are non-local. They provide a means of
communicating the reason for not supplying a service oor the fact that a problem
has been encountered while providing the service to the client who requested
the service.

%-----------------------------------------------

\subsection{What is an Exceptional Situation?}

A server encouters an exceptional situation when it does not know when it cannot
supply the requested service. For example, if you want to withdraw funds from an
account and your account cannot provide the service because there are
insufficient funds in the account or because it has been frozen due to some
legal dispute, then the account object would throw an exception, notifying
you as client that it cannot supply the service.

%-----------------------------------------------

\subsection{What does a client do with an Exception?}

Assume you requested a withdrawel from your account and that the account
did not oblige, but threw an exception instead.

If you ignore the exception, the person who asked you for the cash (you client)
will be notified. In other words exceptions are passed up the client-server
hierarchy (or the call-hierarchy). 

On the other hand you could decide to catch the exeption. When you catch an 
exception you have to provide the exception handling code. Here you could 
specify that if your withdrawel from account A failed that you will try and 
withdraw from account B or that it is time to visit your father in law or 
whatever.

%------------------------------------------------------------------------------

\subsection{Throwing and Catching Exceptions}

We shall first look at the problem detection which occors at the server side
and then we shall focus on the exception handling on the client side.

%------------------------------------------------------------------------------

\subsection{Creating Exception Classes}

In {\em C++} you can throw an instance of any class when an exceptional 
situation arises. Typically you would define a simple class exception class
which may, at times, carry no information beyond the information conveyed 
through its identity. Such a class would simply be defined as an empty class
(of course the compiler will provide default and copy constructors as well as 
an assignment operator and a \verb+this+ pointer).

For example, we could define an \verb+InsufficientFunds+ exception 
simply as follows:

\noindent{\small \begin{verbatim}
class InsufficientFunds {};
\end{verbatim}}

%------------------------------------------

\subsubsection{Exceptions which carry additional information}

Often one would like to add further information to an exception class.
Since any class can be used for exceptions, one can add any attributes
as well as any services. One should however refrain from adding any
attributes  or services which are unrelated to the core purpose of 
exceptions, that of conveying information about the reasons why the
requested service could not be supplied.

For example, we could define an insufficient funds exception which
carries information about the account which raised the exception as
well as the funds available in that account:

\noindent{\small\begin{verbatim}
  class InsufficientFunds
  {
    public:
      InsufficientFunds(Account* const source,
                        const double availableFunds)
         : _source(source), _availableFunds(availableFunds) {}

      Account* const getSource() const {return _source;}

      double availableFunds() {return _availableFunds;}

    private:
      Account* const _source;
      const double _availableFunds;
  };
\end{verbatim}}

%----------------------------------

\subsection{The Server Side: Throwing Exceptions}

The server detects a situation where it cannot complete the requested service
and throws an exception. For example, in the \verb+debit()+ method of our
\verb+Account+ class we check for sufficient funds.

\noindent{\small\begin{verbatim}
class Account
{
  public:
	  ...
	  void debit(double amount) 
	  {
	    if (amount > balance)
	      throw InsufficientFunds(this, _balance);
	    else
	      balance -= amount;
	  }
  ...
}      
\end{verbatim}}

When the problem is detected we instantiate the relevant Exception class and
throw the exception via java's \verb+throw+ keyword. Note that the function
announces that it may throw an \verb+IllegalArgumentException+. This is done
with a \verb+throws+ clause in the method header.

This is all the server does. What could he do more anyway? The control is
transferred at the \verb+throw+ clause from the server to the client -- the
remainder of the method body is skipped.      

Note that when the exception is thrown the server code is exited, i.e.\ the 
remaining statements are skipped.

%----------------------------------

\subsubsection{Exception Notification}

C++ also has support for exception notification. For example, if you want to
notify the clients of the \verb+Account+ class that the debit service is 
not provided unconditionally, i.e.\ that under certain circumstances it 
throws an \verb+InsufficientFunds+ exception, then you add a \verb+throw+
clause to the method header:

\noindent{\small\begin{verbatim}
class Account
{
  public:
  	...
    void debit(double amount) throw (InsufficientFunds);
  ...
}      
\end{verbatim}}

In the above listing we declare that the \verb+Account+ class may raise
an \verb+InsufficientFunds+ exception. In other words, that the \verb+debit+
service is not supplied unconditionally. We are also saying that this
service will NOT raise any other exception.

If a service raises multiple exceptions, these are simply inserted into
the brackets. For example, if we constrain the amount which may be 
withdrawn on a single day from an account, we may specify the \verb+throw+
clause of the \verb+debit+ service as follows:

\noindent{\small\begin{verbatim}
class Account
{
  public:
  	...
    void debit(double amount) throw (InsufficientFunds, DailyLimitExceeded);
  ...
}      
\end{verbatim}}

We may also want to explicitly promise that a particular service is provided
unconditionally. In this case the \verb+throw+ clause is followed by an
empty bracket:

\noindent{\small\begin{verbatim}
class Account
{
  public:
  	...
    void credit(double amount) throw ();
  ...
}      
\end{verbatim}}

%---------------------------------

\subsubsection{Exception Notification Violations}

Assume a method or function promises in a \verb+throw+ clause not to throw
any exceptions or only to throw certain exceptions. What happens if it
in practice does throw another exception. This will only be known at
run-time. The function \verb+unexpected()+ defined in the C++ standard
library which, by default, will terminate the application.

%----------------------------------

\subsection{The Client Side: Catching Exceptions}

The client makes use of the services offered by the server by sending messages
to it. The service may complete successfully or the server may throw an
exception.

The client may or may not be able to handle the exception. If the client is not
able to handle the exception it will be passed up the client server hierarchy
(calling hierarchy) until, hopefully, there is a point where there is enough 
information to handle the problem. 

If a client is willing to catch an exception, it puts the service request message
within a \verb+try+ block:

\noindent{\small\begin{verbatim}
class TheClientClass
{
  public;
    ...
    void aClientMethod()
	  {
	    ...
	    try
	    {
	      account.debit(amount);  // this service may throw an exception
	      ...
	    }
	    catch (InsufficientFunds e)
	    {
	      /* Exception handling code comes here.  
	         Sorry, this is still your baby. */
	    }
	    ...
	  }
};    
\end{verbatim}}

If an exception is thrown control is transferred from the service-request
statement to the corresponding catch clause and, depending on the content
of the catch clause, execution continuous with the statements following 
the catch clause.

Note that if the client makes use of services which may through an exception,
and if the client does not catch the exception, the client's method must
notify users that it may thow that exception, even though it does not do so 
explicitly:

\noindent{\small\begin{verbatim}
class TheClientClass
{
  public:
    ...
    void aClientMethod() throw (InsufficientFunds)
	  {
	    ...
	    account.debit(amount);  // this service may throw the exception
	    ...
	  }
}    
\end{verbatim}}

%------------------------------------------------------------------------------

\subsection{A Complete Example}

Let us now look at the complete code for the \verb+Account+ example:

%----------------------------

\subsubsection{Account.h}

\noindent{\small \input{ErrorHandling/Programs/Account.h}}

%----------------------------

\subsubsection{Account.cpp}

\noindent{\small \input{ErrorHandling/Programs/Account.cpp}}

%----------------------------

\subsubsection{TestException1.cpp}

\noindent{\small \input{ErrorHandling/Programs/TestException1.cpp}}

Running the application yields the following ouput:

\noindent{\small \begin{verbatim}
Debited my account. Balance: 1700
Debited my account. Balance: 1350
Debited my account. Balance: 1000
Debited my account. Balance: 650
Debited my account. Balance: 300
my Account only has R300 left. Debited uncle instead.
my Account only has R300 left. Debited uncle instead.
my Account only has R300 left. Debited uncle instead.
my Account only has R300 left. Debited uncle instead.
my Account only has R300 left. Debited uncle instead.
my Account only has R300 left. Debited uncle instead.
my Account only has R300 left. Debited uncle instead.
my Account only has R300 left. Debited uncle instead.
my Account only has R300 left. Debited uncle instead.
Out of luck. Uncle broke too.
\end{verbatim}}

%------------------------------------------------------------------------------

\subsection{Catching Exceptions at various levels of Abstraction}

In order to be able to handle exceptions at different levels of abstraction
we have to introduce a class hierarchy for exceptions. For example, we may
feel want to introduce the concept of an exception through a base class
\verb+Exception+. We could then define a specialized \verb+TransactionFailed+
exception with further specializetions like \verb+InsufficientFunds+
and \verb+DailyLimitExceeded+ exceptions:

\noindent{\small \begin{verbatim}
class Exception {};

namespace finance
{
  class TransactionFailed: public Exception {};
  
  class InsufficientFunds: public TransactionFailed {};
  
  class DailyLimitExceeded: public TransactionFailed {};
  
  class AuthorizationFailed: public TransactionFailed {};
}  
\end{verbatim}}

Once we have done this we can handle exceptions at different levels of 
abstraction. Consider, for example, the following code:

\noindent{\small \begin{verbatim}
try
{
  account.debit(amount);
}
catch (InsufficientFunds)
{
  /* handle this problem one way */
}
catch (TransactionFailed)
{
  /* Handle all other TransactionFailed exceptions this way */
}
catch (Exception)
{
  /* Handle all other exceptions yet another way. */      
}
\end{verbatim}}

%----------------------------

\subsubsection{Catching all Exceptions}

C++ also supports a \verb+catch+ clause which catches all exceptions
irrespective of whether they are part of the same class hierarchy or
not. To achieve this one inserts 3 dots into the arguments brackets
of the catch clause:

\noindent {\small \begin{verbatim}
try
{
  account.debit(amount);
}
catch (InsufficientFunds)
{
  /* Handle this problem one way. */
}
catch (...)
{
  /* Catch anything else which is thrown in this way. */
}
\end{verbatim}}    

%------------------------------------------------------------------------------

\subsection{Partial Handling of an Exception}

Assume that we want to perform some action upon an exception being thrown by 
a service we requested, but that we cannot handle it completey (i.e.\ cannot
decide how to resolve the problem completely). In such a situation we can
catch the exception, perform some action in the \verb+catch+ block and either
rethrow the same exception or throw another exception from within the catch
block. In this way we notify the users of our method that we still have not
resolved the problem completely.

C++ has a special syntax for rethrowing the same exception:

\noindent {\small \begin{verbatim}
try
{
  /* do some IO */
}
catch (EndOfFile)
{
  /* Process data. */
}
catch (...)
{
  /* Close files and rethrow. */
  
  ifsteam.close();
  
  throw;
}
\end{verbatim}}

%================================================================

\section{Throwing Type Declarations}

C++ also allows you to throw primitives like enumerated integers:

\noindent {\small \begin{verbatim}
class Account
{
  private:
    enum State {success, noFunds, authorizationFailed, limitExceeded};
    State state = success;
    
  public:
    void debit(amount)
    {  
      ...
      if (amount > _balance)
        state = noFunds;
      ...    
      if (state != success)
        throw state;
      ...
   }
};
\end{verbatim}}

%-------------------------------------------------------------------

\subsection{Defining Exception Classes as Nested Classes}

It is often a good idea to package the exception class together
with the class which may raise the exception.

For example for a {\bf Rational} class developed may raise
a {\bf DivideByZero} exception if the denominator of a
{\bf Rational} object becomes zero and a \verb+OutOfRange+
exception if the numerator or denominator falls outside
the range of the \verb+long+ data type:

\noindent {\small \begin{verbatim}
class Rational
{
  public:
    Rational (const long int Numer, const long int Denom) ;
    ...
    long int numerator();
    long int denominator();
    ...
    Rational operator + (const Rational& r)
    ...
    class DivideByZero {};     // exception class for /0 error
    class OutOfRange   {};     // exception class for range error
  private:
    long int numer, denom;
}
\end{verbatim}}

The calling program could catch these exceptions in the following way

\noindent {\small \begin{verbatim}
void main()
{
  ...
  try
  {
    long int a, b, c, d;
    ...
    calc1(a,b);
    calc2(c,d);
    ...
    Rational r1(a,b), r2(c,d);
    Rational r3 = r1+r2;
  }
  catch (Rational::DevideByZero)
  {
     ...       // error handler for /0 errors
  }
  catch (Rational::OutOfRange)
  {
     ...       // error handler for range errors
  }
  ...
}
\end{verbatim}}

%-----------------------------------------------

%\section{Exercises}
%\begin{enumerate}
%  \item 
%\end{enumerate}
