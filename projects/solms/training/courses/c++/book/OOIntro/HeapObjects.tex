\section{Creating Objects on the Heap}

So far we created objects on the stack. The objects were
scoped to within a block, often a function body (e.g.\ \verb+main+) and
are stored within the stack frame of that object. Objects
which have been declared on the stack are automatically 
deleted when they go out of scope.

However, the stack is a limited resource. Furthermore, one often requires
objects to survive the scope in which they have been created. More often
than not, one would want to create objects on the heap. This requires,
however, that the developer has to control the memory management of the
object and that introduces considerable risks in terms of potential
memory losses as well as dangling pointers. The latter happens if an
object is deleted while another part of the application still has a
pointer to it. At a later stage this pointer could be used resulting
in system corruption or system crash.

This problem is non-trivial -- so much so that many commercial C++ applications
end up with memory leaks. In fact, there is a market for C++ memory leak 
detector tools. From a more purist perspective the memory management is best
tackled through a solid design (perhaps using UML) whereeach object has 
ultimately one owner who takes over the memory and pointer management for
that object.

Objects are created on the heap via the new operator. They are deleted via
the delete operator which ultimately calls the destructor. Below is an
application which creates a collection of accounts on the heap, sends them 
through to a function and finally deletes them:

\noindent {\small \input{OOIntro/Programs/Account6.cpp}}

