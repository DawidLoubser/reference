\section{Objects and Classes}

In object-oriented programming the central concept is obviously an object.
Let us first clarify the concepts of objects and classes, before going 
into the C++ language syntax for defining classes adn creating and using
objects.

%----------------------------------------------------------------------------

\subsection{What is an Object?}

The central concept of object-oriented programming is an object. An object is
a unit which
\begin{itemize}
  \item has identity,
  \item has attributes and
  \item can perform operations.
\end{itemize}  

We think object-oriented. When we use a nouns in a sentence structure we
generally refer to objects. For example,

\centerline{\em ``The green car drives along the lane.''}

Here the nouns, {\em car} and {\em lane}, are objects. The objects have attributes, 
i.e.\ the car is {\em green}, and can perform operations, \i.e.\ the car {\em drives}.
Hence, object orientation is not an unfamiliar concept in human thought -- it is our
natural way of thinking.

%----------------------------------------------------------------------------

\subsection{Identifying Objects}

We have already mentioned one way to identify objects. That is by extracting the nouns 
from a linguistic description of the system we are modeling. Otherwise we can try
and identify units which have a clear conceptual meaning and which can be assigned
an identity which makes them unique. Typically these units would have attributes
and typically they can conceptually perform certain operations, i.e.\ it is natural
to give thjem responsibilities for certain tasks.

Another approach is to think of a system and identify its components. Each of the 
components would be an object in their own right. In this way you can go on to
deeper and deeper levels of components.

Alternatively you can look at an object and identify any messages it sends in order
to request certain services (i.e.\ identify the recipients of the function calls).
The recipients of these messages would themselves be objects.

%----------------------------------------------------------------------------

\subsection{Classes as Abstractions of Objects}

A class is a template from which objects are created. It encapsulates the commonalities
of all instances (objects) of that class. Objects are instances of classes.

A class defines the attribute types as well as the operations (services) which 
its instances have. For example, an account may have an account number, a reference
to an owner, a balance and may offer services for crediting, debiting and
querying the balance. A particular account is an instance of the class. The
class itself is a more abstract concept. One can discuss features of the
class of accounts which apply to all account instances.

%-------------------------------------------------------------------------

\section{Defining Classes}

A class is a template from which objects are generated. Conceptually
one defines in a class the commonalities (common attributes and services)
of the instances of the class. 

%------------------------------------------------------------------------

\subsection{A Simple Account Class \label{secSimpleAccount}}

Below we show a simple account class which specifies that instances of
that class can be credited and debited and that the balance can be queried:

\noindent {\small \input{OOIntro/Programs/Account1.cpp}}


%------------------------------------------------------------------------

\subsection{Methods/Services}

A Method interface or header specifies how users can interface with the method,
i.e.\ how to call the method. The interface of a method
is defined by the name of the method, the argument types and the type of
the return value if any. The syntax for a method interface definition is

\noindent{\footnotesize \begin{verbatim}
<ReturnType> <methodName>(Argument1Type arg1Name, Argumet2Type arg2Name, ...)
\end{verbatim}}

%----------------------------

\subsubsection{Method names}

The method name should start with a lower case letter with word boundaries 
capitalized. Otherwise the same rules hold as those for all other identifier
names (see section \ref{secIdentifiers}).

%----------------------------

\subsection{Access levels}

For the time being we shall look at only two access levels, \verb+public+
and \verb+private+. The former specifies that a class member is publically
accessible from anywhere. The latter specifies that a member can only be 
accessed from within the class itself. We shall return to the issue of
access levels in section \ref{secAccessLevels}.

In C++ an access level is specified for a blocks. The access level applies
for all consecutive elements until it is changed.

\noindent{\small \begin{verbatim}
class Account
{
  public:
     void debit(double amount);
     void credit(double amount);
     ...
     
  private:
    double balance;
};
\end{verbatim}}       

%------------------------------------------------------------------------

\subsubsection{Encapsulation}

The \verb+public+ elements of a class are those which are meant for
general use. They typically include the public services offered by
instances of the class.

All the implementation details should be encapsulated within the class,
i.e.\ should not be visible from outside the class. This ensures that
users of the class don't develop dependencies on these implementation
details.

Consider the following scenario: Assume you wrote a \verb+Date+ class
with 3 public data fields:
\noindent{\small \begin{verbatim}
class Date
{
  public:
    increment();
    addDays(int numDays);
    ...
 
  public: 
    int day, month, year;
};  
\end{verbatim}}

Of course, the \verb+increment()+ and \verb+addDays(numDays)+ methods
are non-trivial, having to check for end-of-year, end-of-month and
potentially for leap years and after using significant amounts of code
of the form

\noindent{\small \begin{verbatim}
for (Date d = d1; d<d2; d.increment())
   ...;
\end{verbatim}}

\noindent
it may seem an excessive burden to do all this work for every increment.
At some stage you might come to the conclusion that your for-father's
idea of using the earth's rotation around its own axis, the moon and 
the earth's rotation around the sun as reference was not such a good idea
after all and you may decide to work with days in some new units. 
Keeping still the earth's rotation around it's own axis, you may decide
to store a date as an absolute date, choosing your favourite date (e.g.\
the first time you came home after midnight and youre parents actually
accepted it) as day number 1 and counting days sequentially. After 2 years
you'll reach day 730 and you were born on day -5475. Now, incrementing 
and decrementing dates or adding a number of days to a date and many other
functions become trivial and the only time you are faced with any of the 
old complexity is when you have to convert between your snazy internal
units and the archaic day, month, year units of your for-fathers. 

If your data fields (day, month, year) had been declared private to start 
off with, you could simply make the implementation changes to your class
and that would be it. If, on the other hand, they were declared public,
they may be accessed from any other code distributed throughout your 
organization and you would have to search for sny such code and make
the corresponding changes there.

Furthermore, you cannot guarantee the integrity of instances of your class
if you give public access to your data fields. Anybody could go ahead
and set the month to 27, bypassing any form of sanity checking you may
havein your set-methods.

%------------------------------------------------------------------------

\subsection{Constructors \label{secConstructors}}

Constructors are used to construct (create) instances of classes -- i.e.\
objects. A constructor has the name of the class and no return value --
not even \verb+void+. You can give a constructor as many arguments as you 
like and you can write multiple constructors, each with different arguments.
Below we added two constructors to our account class. The first one takes no
arguments (the default constructor) and the second one takes the initial 
balance as argument.

\noindent {\small \input{OOIntro/Programs/Account2.cpp}}

%----------------------------

\subsubsection{Member Initialization via the Constructor's Parameter List}

In the second constructor the private data field, \verb+_balance+, was
initialized within the body of the constructor. Alternatively we could have 
initioalized the private data field from the argument in the parameter
list like this:

\noindent{\small\begin{verbatim}
class Account
{
  public:
    ...

    Account(double balance): _balance(balance) {}
    ...
}    
\end{verbatim}}

This is effectively done in the first, the default constructor. If a new
account is created it is automatically credited by 50 whatevers.

%----------------------------

\subsubsection{Compiler-Generated Default Constructor}

Recall that in our earlier \verb+Account+ class example (see section
\ref{secSimpleAccount}) we did not define a constructor at all. Yest, 
we were able to create accounts via

\noindent{\small\begin{verbatim}
Account acc1;
\end{verbatim}}

How was this possible? The compiler wrote a default constructor -- one
which takes no arguments and has an empty body -- for us. However, if
we had added a constructor which takes arguments (e.g.\ the one which
takes the initial balance as argument) without adding a default 
constructor, we can no longer create an account in this way:

\noindent {\small \input{OOIntro/Programs/Account3.cpp}}

The reason for this is that C++ (like Java) creates a default constructor 
for you if and only if you did not define any constructor whatsoever
for your class. In general you should write your own default constructor
if you want a default constructor and disable it if you don't. If you want 
a class without any constructor whatsoever, you can define a \verb+private+
default constructor.

%------------------------------------------------------------------------

\subsection{The OO Naming Convention}

We have adhered to the OO naming convention which is very simple, clean
and effective. It is used throughout the C++ community (except in the
Microsoft community), in Java, Smalltalk and UML (the Unified Modeling
Language) and simply states:
\begin{itemize}
  \item Class names start with capital letter. Word boundaries are
        capitalized.
  \item Everything else (variable names, method names, function names and
        object names) start with lower case letter. Word boundaries are 
        still capitalized.
\end{itemize}

%----------------------------------------------------------------------

\section{Requesting services from objects}

In section \ref{secConstructors} we requested several services from
different accpount instances (objects).
In object-orientation one requests a service from an object by
sending a message to it. One does not call a function -- we shall 
see later that the client often does not know which actual function
is called.

To send a message to an object one uses the member selection operator,
the dot. For example, if we want to send a message to an account asking 
for the balance we can do it as follows:

\noindent{\small\begin{verbatim}
double bal = acc1.balance();
\end{verbatim}}

Similarly, if we want to debit that same account, we can send a debit
message:

\noindent{\small\begin{verbatim}
double bal = acc1.debit(500);
\end{verbatim}}

%-------------------------------------------------------------------------

\section{The Life-Span of an Object on the Stack}

So far we have created objects on the stack, i.e.\ they life in the
same stack frame as other local variables. An object on the stack then
also has a life span similar to that of local variables. It exists
from where it is declared until the end of the block (the closing
curly bracket) in which it is declared.

%--------------------------------------------------

\subsection{Destructors}

Destructors are called when the object is deleted. Destructors are 
responsible for releasing any memory which the class grabbed from the
heap as well as releasing other resources like closing files or network
sockets. We shall see that when
objects are created on the heap, the destructor has to be called manually.
However, for objects which have been created on the stack, the destructor
is called automatically.

Like constructors, the destructor also carries the name of the class, but
it is preceded by a tilda and it may not have any arguments. The following
example shows a destructor which simply laments the death of an object and
illustrates how the destructor is called automatically as the object leaves
its scope.

\noindent {\small \input{OOIntro/Programs/Account4.cpp}}

The output of the application looks something like this:

\noindent{\small\begin{verbatim}
Entered f().
About to leave f()
I, 0012FF44, am destroyed.
Created acc1, entering block.
Created acc2 in block,leaving block.
I, 0012FF7C, am destroyed.
\end{verbatim}}