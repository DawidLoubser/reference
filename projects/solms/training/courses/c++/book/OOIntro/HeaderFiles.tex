\section{Splitting Headers and Implementation}

Naturally one does not want to define all code (classes and functions) within a
single source file. Furthermore, one does not want to recompile the entire
source if one makes a modification to a specific area of the source.

To this end C++ enables you to define the header of a class which contains the
method headers and the data fields of a class separate from the implementation
code of these methods. For example, we can define the header of the verb+Account+
class in a file \verb+Account.h+ and its method implementations in a file called
\verb+Account.cpp+.

%-------------------------------------------------------------------------------

\subsection{The Header File}

The header file defines the method headers without the method bodies. Below we
list a C++ header file for our account class:

\noindent{\small \input{OOIntro/Programs/Account.h}}

Note that macros are used to avoid duplicate inclusion of the header file. A macro
variable, \verb+ACCOUNT_H+ is defined the first time the file is read. The contents 
of the header file is only included if the variable has not yet been defined within
the compilation process. 

%-------------------------------------------------------------------------------

\subsection{The Implementation File}

The implementation file defines the method bodies of the methods whose header is 
specified in the header file. Since header and implementation files may contain 
multiple classes, the scope of the function must be specified. For example

\noindent{\small\begin{verbatim}
Account::debit(double amount) {_balance -= amount;}
\end{verbatim}}

\noindent
defines the body of the \verb+debit+ method of the \verb+Account+ class. Similarly,

\noindent{\small\begin{verbatim}
Account::Account(): _balance(50) {}
\end{verbatim}}

\noindent
defines the implementation of the default constructor. The complete implementation
file is listed below:

\noindent{\small \input{OOIntro/Programs/Account.cpp}}

%-------------------------------------------------------------------------------

\subsection{Including Header Files}

C++ supports tw notations for including header files -- they may be either specified
within tag delimiters or within quotes. The former refers to header files which can 
be located along the system path while the latter notation is used for files which
are located relative to the current directory (using relative paths) or at specified
locations (using absolute paths). 

Our main program includes the \verb+Account.h+ header file as well as some system
header files:


\noindent{\small \input{OOIntro/Programs/Account5.cpp}}

%-------------------------------------------------------------------------------