\section{Class Members}

So far most of the members we added to our classes (methods and data fields)
were instance members. For example, each instance of the account class had
its own \verb+_balance+ and when we requested the \verb+debit(double)+, 
\verb+credit(double)+ or the \verb+balance()+ query service, we sent the
message to a specific instance (object):

\noindent{\small\begin{verbatim}
account1.credit(550);

cout << account1.balance();
\end{verbatim}}

The only members which were class members thus far were the constructors.
They had the name of the class and where implicitely class services. When
we create an account we requested the service from the \verb+Account+ class 
and not a specific instance of the class -- a particular account. 

At times one wants to assign also other responsibilities to the class itself. For
example, we may want to keep track of the number of instances of the account
class. This responsibility does not fit naturally within a specific instance
of the class. Instead, we might want to assign this responsibility to the
class itself -- after all, the class is responsible for creating instances
(if one uses a factory pattern, the responsibility is naturally hosted by
the factory).

In C++, as in Java, one uses the keyword \verb+static+ to specify that a
specific element is a class member. For example, we may want to assign a
datafield \verb+_numInstances+ to the class as well as a query service,
\verb+numInstances()+. We would declare both these members \verb+static+,
i.e.\ class members. Below we show the header file of a \verb+Client+
class which keeps track of the number of instances of that class:

\noindent{\small \input{OOIntro/Programs/Client.h}}

Nte that the class itself hosts a data field, \verb+_numInstances+, which can 
be accessed from within the class scope or from the within the scope of any
particular instance of the \verb+Cient+ class.

%-----------------------------------------------------------------------

\subsection{Constructors are Class Services}

One uses the keyword \verb+static+ to specify class services and datafields
of the class. However, constructors themselves are also class services --
one asks the \verb+Client+ class for an instance, not a particular
account (there may not even be an account instance around yet).
So constructors are implicitly static methods. 

In our constructor we added a line which increments the instance counter:

\noindent{\small\begin{verbatim}
Client::Client(char* name)
{
  ++_numInstances;
  _name = name;
}
\end{verbatim}}

%-----------------------------------------------------------------------

\subsection{Destructors are Instance Services}

Destructors are not class services. Their resposibility is to provide some
finalization code (e.g.\ clean-up code) when an object (a particular
instance) is deleted. The message is sent to a specific \verb+Client+.
Still, the instance counter should be decremented:

\noindent{\small\begin{verbatim}
Client::~Client()
{
  --_numInstances;
}
\end{verbatim}}

Note that class members can be directly accessed from within instance members 
(like the destructor), but not vice versa.

%-----------------------------------------------------------------------

\subsection{Specifying Implementation Details of Class Members}

The body of a static method is defined in the same way as the body of an 
instance method. Naturally, one cannot access instance members from a class
service. After all the scope is the class itself -- which instance should
it refer to. Below we show the trivial implementation of the 
\verb+numInstances()+ query method:

\noindent{\small\begin{verbatim}
Client::numInstances()
{
  return _numInstances;
}
\end{verbatim}}
We still have to initialize the class member, \verb+_numInstances+ to zero. But were 
should we do that? We cannot initialize the data field in the constructor
because then the field would be initialized every time an instance is created.
We want to do once off initialization when the application is loaded. This 
initializtion statement is done at global scope via:

\noindent{\small\begin{verbatim}
int Client::_numInstances = 0;
\end{verbatim}}

The full implementation file, \verb+Client.cpp+ is listed below:

\noindent{\small \input{OOIntro/Programs/Client.cpp}}

%-------------------------------------------------------------------------

\subsection{Using Class Members}

Off course, we have used some of them already -- the cnstructors. Below we
show an example application which creates a few accounts, uses them and then 
deletes them. In between we ask the class now and then to report the number
of instances which currently exist for the class:

\noindent{\small \input{OOIntro/Programs/ClientTest.cpp}}

