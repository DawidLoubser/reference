\section{C++ and Interfaces}

Interfaces have become a very important concept in the software world.
Not only do the Unified Modeling Language (UML) and Java directly
support the concept of an interface, but a lot of the progress in the
software development industry has been around interfaces.

For example, the entire CORBA specification published by the OMG is
only an interface specification. So is the EJB framework published by
Sun. Most new class libraries are designed around interfaces. These include
the Standard Template Library of C++, the Java Messaging, Transaction and
Services (JMS and JTS) and many others.

An interface simply specifies a set of services which classes which
implement the interface must supply. As such an interface is like
an abstract class with only abstract methods -- and, yes, this is the
way you implement interfaces in C++.

For example, you may specify that any class which claims to be a source
of interest rates should implement an \verb+InterestRateSource+ interface
which requires the implementing classes to supply two services:

\noindent{\small \begin{verbatim}
class InterestRateSource
{
  public:

    virtual InterestRate getRate(const Date& date1, const Date& date2) const = 0;

    virtual double getDiscountFactor(const Date& date1, const Date& date2) const = 0;
};
\end{verbatim}}

A wide range of classes could potentially supply these services. Here we
list a few in order to illustrate how totally different their implementations
can be:
\begin{description}
  \item[ReutersRateSource] which reports the current market rates as perceived
        by Reuters.
  \item[ZeroCurve] which stores spot interest rates for investements of
        different durations made at the spot date. This class would
        calculate interest rates over periods covered by the curve from
        the spot rates.
  \item[DealerWindow] which pops up on the terminal of a dealer on the
        interest rate desk. He/she is expected to know the market and
        would be in a position to supply the interest rate for a supplied
        period.
  \item[BondPortfolio] which contains a collection of priced Bonds. 
        Interest rates for  a given period can be calculated from the
        bond prices.
\end{description}

These different implementation have virtually nothing in common, except, that
they all can provide the service. But why should you lock within your code
into a specific service provider. If your code couples to the interface instead,
you will be able to plug in any implementation of an interest rate source.
You are thus decoupling from specific implementations and have the basics
of plug-and-play programming.


