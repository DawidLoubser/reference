
%------------------------------------------------------------------------------

\section{Abstract Classes}

So far all our classes were {\em concrete}, i.e.\ we were able to create
instances of them. At times one wants to introduce classes without having 
the intention of actually instantiating them. The reasoning may be 
that one still wants to encapsulate some commonalities accross classes 
within a single superclass, but that the class which encapsulates these
commonalities by itself. Such a class would be an {\em abstract} class.

For example, all assets have, say an id. Specialized assets like properties,
vehicles, government or corporate bonds, ..., can be created, but one
would never create just simply an asset. Still, one would want to define
a \verb+Asset+ class encapsulating all the commonalities among all assets
(the asset id in our case). The asset class would be, conceptually,
an abstract class.

%----------------------------------------------------------------

\subsection{How to Declare a Class with Concrete Methods Abstract}

C++, unlike Java, does not have a mechanism for declaring a class directly 
abstract. What we need to achieve is that the class can only be instantiated
from within the context of one of its subclasses. But this can be done
simply by declaring its constructors with \verb+protected+ access level:

\noindent{\small\begin{verbatim}
class Asset  
{
  public:
    string id() const {return _id;}
    
  protected:
    Asset(const string& id) _id(id) {}
    
  private:
    string _id;  
};
\end{verbatim}}

Abstract classes usually have at least one concrete subclass, i.e.\ one 
which can be instantiated. For example

\noindent{\small\begin{verbatim}
class Bond: public Asset
{
  public:
    Bond(string id, Date maturity, double notional, ...);
    ...
}
\end{verbatim}}
  
%------------------------------------------------------------------------------

\subsection{Abstract Methods for Interface Definition}

One of the main advantages of abstract classes is that they can incorporate
interface specifications via abstract methods. Assume, for example, that
all assets can be queried for their value on a specific date. How the 
value of an asset is calculated depends on the type of asset. 
We want to specify, 
however, that all assets have a service for querying the value, i.e.\
that they all must be able to process a \verb+getValue(Date)+ message. 
This is donme in C++ by adding an abstract method -- i.e.\ a method for
which there is no implementation -- to the class:

\noindent{\small\begin{verbatim}
virtual Tender value(const Date& date) = 0;
\end{verbatim}}

\noindent
Of course, we do not know how to value an 
abstract asset. An abstract method is one which has no method body. A
class which has one or more abstract methods is implicitly an abstract
class and can hence not be instantiated. 

%-------------------------------------------------------------------------

\subsection{SubClassing Abstract Classes}

Subclasses of abstract classes must either comply to the specifications
laid down in the abstract superclass (i.e.\ provide implementations for 
the abstract methods of the superclass) or they must be declared abstract
themselves. 

For example, subclasses of the abstract \verb+Asset+ class must provide
an implementation for the \verb+value(Date)+ service. If they don't, the
compiler will regard the subclass itself still as abstract. The reason 
for this is that the subclass does not fulfill the requirements
laid down in the superclass. Since an abstract class cannot be instantiated,
the compiler ensures that there will be no \verb+Asset+ objects (instances
of the \verb+Asset+ class) which cannot be valued.

%------------------------------------------------------------------------

\subsection{Abstract, Virtual and Non-Virtual Methods}

There are three types of service specifications you can define in
a class which may be subclassed:
\begin{itemize}
  \item Pure specifications for services which must be supplied by
        concrete subclasses.
  \item Services with default implementations which may be overridden.
  \item Services with a fixed implementation which you should NOT override.
\end{itemize}
Each of these is dicussed seperately below.

%------------------------------
  
\subsubsection{Abstract or pure virtual methods}

Abstract methods lay down specifications  for services which must be supplied 
by concrete subclasses. No implementation (body) is supplied for abstract 
methods -- the implementation must be supplied by the concrete subclasses. 
In this case one inherits only a requirements specification.

An abstract method must be declared \verb+virtual+ and is specified by
an \verb+ = 0 + behind the message header:

\noindent{\small\begin{verbatim}
virtual Tender value(const Date& date) = 0;
\end{verbatim}}

Abstract or pure virtual methods are allways hosted by abstract classes.

%--------------------------------

\subsubsection{Concrete virtual methods} 

Non-abstract virtual methods still require subclasses to provide
the specified service, but if they do not supply their own
implementation, a default implementation is provided by the 
superclass. This default implementation may, but need not, be 
overridden in the subclass.

A concrete virtual method is thus specified with an implementation.
For example, an account class may have a virtual \verb+debit+ method
specifying that all accounts must supply such a service and providing
a default implementation in which the balance is simply adjusted with
the debit amount.

Specialized subclasses like \verb+CreditCard+, \verb+ChequeAccount+
or \verb+HomeLoan+ may override this method with their specialized
implementation (for example, subtracting a transaction fee), but they
may also choose to simply inherit the default implementation.

A concrete virtual method is simply a \verb+virtual+ method with a
supplied function body:

\noindent{\small\begin{verbatim}
virtual void debit(double amount) { _balance -= amount;}
\end{verbatim}}

%-----------------------------------

\subsubsection{Non-virtual methods}

Non-virtual methods should be regarded as methods with a fixed 
implementation which subclasses should NOT override. C++ does 
not prevent the method from being overridden, but if you do, 
you may get unexpected behaviour. They are specified as methods
without the \verb+virtual+ keyword. 

For example, we might want to specify that the \verb+authenticate+ 
method should not be overridden in specialized \verb+User+ classes,
i.e.\ no matter how special the user, when he is authenticated the
block of code specified in the \verb+User+ class should be executed:

\noindent{\small\begin{verbatim}
class User
{
  public:
    ...
    void authenticate() { ... }
    ...
};
\end{verbatim}}
    
Non-virtual methods are linked statically (i.e.\ at compile time). 
They do not support polymorphism and should only be used if you want to
specify that that particular method may not be overridden. By default, you
should define your methods \verb+virtual+.

So, why is it so bad to override non-virtual methods. Simply because your
system may exhibit strange, unexpected behaviour. The problem is 
illustrated in the following little program where we have an instance of
a class \verb+A+ providing a virtual method \verb+f()+ and a non-virtual
method, \verb+g()+. Both methods are overridden in a subclass, \verb+B+.

\noindent{\small \input{Abstraction/Programs/OverrideNonVirtual.cpp}}

The poutput of the application is

\noindent{\small\begin{verbatim}
Requesting virtual service f():
  We request the same service from the same object through
  two different message paths (two different pointers.

Requested virtual service f() from an instance of B.
Requested virtual service f() from an instance of B.

Behaved as expected. Now requesting non-virtual service g():
  Once again we request the same service from the same object
  through two different message paths (two different pointers.

Requested non-virtual service g() from an instance of A.
Requested non-virtual service g() from an instance of B.

This time the object behaved differently, depending on the
message path (pointer) used to deliver the service request message.
\end{verbatim}}

Note that we create an instance of \verb+B+ and have an \verb+B*+ and and 
\verb+A*+ pointing to this one and only object. We then request the service 
\verb+f()+ through both of these pointers, and, as expected, \verb+B::f()+
is called in both cases. After all, we only have a \verb+B+.

We then request the non-virtual service \verb+g()+ through both of these 
pointers, but now \verb+B+'s service \verb+g()+ is supplied when we request
the service through the \verb+B*+ and A's \verb+g()+ when the service
is requested through the \verb+A*+. We never requested the service from
an \verb+A+ though -- there isn't even an \verb+A+. We requested the same
service from the same object and it does different things. This is not 
behaviour we want to see in clean object-oriented systems, and hence
a non-virtual method should not be overridden.