\chapter{Packaging via Namespaces \label{chapNameSpaces}}

Namespaces in C++ perrform the same function as packages in UML
or Java and modules in CORBA. They provide a mechanism for hierachical
naming. But why do we need this?

%-------------------------------------------------------------------

\section{NameSpace Pollution}

So far we declared all our classes and functions within global scope.
One can also define objects and variables at global scope, though we
have refrained from doing this.

One of the problems with declaring elements at global scope is that
one has to somehow ensure that eac name is globally unique. If one
used two libraries which both define a verb+Date+ class at global scope
-- this would not be that unlikely -- then the compiler could not 
distinguish between the two \verb+Date+ classes if both libraries
were used simultaneously.

%-------------------------------------------------------------------

\subsection{Unique Naming}

Prior to
the ANSI/C++ standard which introduced the concept of namespaces, the
approach was to insert a prefix (or append a postfix) into each name,
hoping that this aproach would result in unique names and avoid
the polution of the global namespace. For example, if we defined a
\verb+Date+ class, we could have called the class

\noindent {\small \begin{verbatim}
class SolmsTraining_Date
{
  ...
};
\end{verbatim}}

But this may not be enough to avoid name clashes. Particularly in large
organizations, where there are multiple development teams, one might
want to define further nested scopes for the individual development 
teams. 

For example, both, the front-end and back-end development teams
may define their own date/time class, \verb+Date+. In this case
one may ant to introduce a further level in the naming:


\noindent {\small \begin{verbatim}
class OrganizationName_FrontEnd_Date
{
  ...
};
\end{verbatim}}

\noindent {\small \begin{verbatim}
class OrganizationName_BackEnd_SolmsTraining_Date
{
  ...
};
\end{verbatim}}

The resultant names become very long and the code becomes less and 
less readable. 

%------------------------------------------------------------------

\section{Defining NameSpaces}

A namespace is in C++ a conceptual organization of the global 
namespace into a hierachical structure which is conceptually
very similar to UML or Java packages. As with Java, one should
always put ones classes and functions into namespaces. 

To put a elements into a namespace one uses the \verb+namespace+
keyword:

\noindent{\small \begin{verbatim}
namespace SolmsTraining
{
  const double PI = 3.14159265359;
  
	template <class T> bubbleSort(T array[], const int length) { ... }
	
	template <class T> class Matrix
	{
	   ...
	};
}
\end{verbatim}}	

%-------------------------------------------------------------------

\subsection{Namespaces are Cumulative}

If we define the same namespace at different locations, the contents
of the namespace will be the sum total of all the elements defined
at the various locations for that namespace. Thus

\noindent{\small \begin{verbatim}
namespace SolmsTraining
{
  const double PI = 3.14159265359;
}

namespace TimLewis
{
  void convertGifToJPeg(istraem& gifStream, ostream& jpegStream) {...};
}  
  
namespace SolmsTraining
{  
	template <class T> bubbleSort(T array[], const int length) { ... }
	
	template <class T> class Matrix
	{
	   ...
	};
}
\end{verbatim}}	

would assign the same 3 elements to the \verb+SolmsTraining+ namespace
as the previous example.

%------------------------------------------------------------------

\subsection{Namespaces Span Accross Files}

As with Java packages, namespace elements are typically defined in multiple
files. However, unlike Java's package mechanism, C++ implies no mapping 
between the namespace structure and the directory hierarchy. 

Conversely, a file may contain elements assigned to multiple namespaces.

%------------------------------------------------------------------

\section{Using Elements Defined in Other NameSpaces}

You can either access an element from another name space through a fully
qualified name or by importing it via a \verb+using+ clause. For example,
if we wanted to refer to a matrix class generated from the \verb+Matrix+
template defined in the \verb+SolmsTraining+ namespace, we could do sovia

\noindent{\small \begin{verbatim}
SolmsTraining::Matrix<double> m(20,20);
\end{verbatim}}

\noindent
or we could import the name into the current name spece via the following using
statement:

\noindent{\small \begin{verbatim}
using SolmsTraining::Matrix;
using SolmsTraining::PI;

Matrix<double> m(20,20);

m[0][0]= 2.1*PI;
\end{verbatim}}

%------------------------------------------------------------------

\subsection{Importing Entire Namespaces}

At times you want to have direct access to all elements defined in a 
namespace without importing the elements individually. This is 
particularly useful when reworking existing code which did not make use
of namespaces to use elements which have now been packaged within 
namespaces. To import an entire namespace -- i.e.\ all elements from
that namespace, one adds the \verb+namespace+ keyword after the
\verb+using+ keyword:

\noindent{\small \begin{verbatim}
using namespace SolmsTraining;

Matrix<double> m(20,20);

m[0][0]= 2.1*PI;
\end{verbatim}}

%------------------------------------------------------------------

\section{Hierarchical Naming via Nested NameSpaces}

One would typically want to use a hierarchical naming along development
team boundaries or, more sensibly, along application domain
boundaries. Thus, to encourage re-use accross projects and
accross team boundaries, the packaging should not be done along
either of these criteria. Instead you would want to have a
packaging hierarchy which packages related objects together,
irrespective of the project they have been developed for or
the team which developed them.

An example of elements from a packaging hierarchy which follows the 
along the lines of a clean conceptual structure is shown below:

\noindent{\small\begin{verbatim}
SolmsTraining::Maths::LinearAlgebra::Vector<T>;
SolmsTraining::Maths::LinearAlgebra::Matrix<T>;
...
SolmsTraining::Maths::Numeric::Integrate::SimpsonIntegrator;
SolmsTraining::Maths::Numeric::Integrate::RombergIntegrator;
...
SolmsTraining::Utils::DateTime::Date;
SolmsTraining::Utils::DateTime::TimePeriod;
SolmsTraining::Utils::DateTime::AnchoredTimePeriod;
\end{verbatim}}

Such a hierarchy is conceptually clean and with the support of a
decent documentation generation tool it can simplify the search 
for components -- it facilitates a natural search guided by once 
requirements.

For globally unique naming one may want to prefix
the namespace hierarchy with the inverted domain name of the
organization, e.g.\ \verb+za::co::SolmsTraining::Maths+.

%------------------------------------------------------------------

\section{Splitting Implementation from Header}

When splitting the implementation from the header by writing seperate
\verb+cpp+ and \verb+h+ files, one should keep in mind that within the 
implementation file one has to do one of the following:
\begin{itemize}
  \item Import the names for the elements whose implementation is
        specified in the file.
  \item Import the entire namespace(s). 
  \item Use fully qualified names.
\end{itemize}  

%------------------------------------------------------------------

\section{Anonymous Namespaces}

At times one wants to hide certain elements within a single file.
Such elements are used by other elements in the file but should
are not generally useful. Furthermore, we want to ensure that the 
name used for these elements is save from name-clashes with 
similarly names elements defined in other files. 

In this case one may want to introduce an anonymous namespace. This 
is in some ways similar to Java's \verb+package+ access-level 
specification, just that here the access is restricted to a particular
file:

For example, when writing a collection of sorting routines we may 
want to define a \verb+swap+ function and we may want to ensure that 
the this function is distinguished from any other swap function which
may defines in another library we are using. Furthermore, we do not want
to publish this function for general use because we want to be at 
liberty to modify it as required by the sorting algorithms it serves.

In this case we may want to put our \verb+swap+ function into an 
anonymous namespace as follows:

\noindent{\small \begin{verbatim}
namespace 
{
  template <class T> void swap (T& x, T& y);
}

template <class T> void bubbleSort(T* array);
template <class T> void mergeSort(T* array);
template <class T> void quickSort(T* array);
\end{verbatim}}

%-------------------------------------------------------------------

\section{An Example}

Let us look at an example where we put an account class and a client 
class into seperate namespaces and at a little example application 
which uses these two classes.

%----------------------------------

\subsubsection{Account.h}

The account class is put into a nested namespace, 
\verb+SolmsTraining::finance+:

\noindent{\small \input{NameSpaces/Programs/Account.h}}

%----------------------------------

\subsubsection{Account.cpp}

In the implementation file we use fully qualified names to resolve
the packaged items:

\noindent{\small \input{NameSpaces/Programs/Account.cpp}}

%----------------------------------

\subsubsection{Client.h}

Similarly, the client class is put into a nested namespace, 
\verb+SolmsTraining::clients+. It uses the \verb+Account+ class
and avoids having to use fully qualified naming for it by importing
it via a \verb+using+ satement:

\noindent{\small \input{NameSpaces/Programs/Client.h}}

%----------------------------------

\subsubsection{Client.cpp}

In this implementation file we totally avoid using fully qualified
names by importing both, the \verb+Account+ class from the 
\verb+SolmsTraining::finance+ namespace and the \verb+Client+ class
from the \verb+SolmsTraining::clients+ namespace:

\noindent{\small \input{NameSpaces/Programs/Client.cpp}}

%----------------------------------

\subsubsection{TestNameSpace.cpp}

Finally we use these classes in the following example application:

\noindent{\small \input{NameSpaces/Programs/TestNameSpace.cpp}}

%-------------------------------------------------------------------

\section{Finding Classes}

Packaging can help significantly in finding classes. One should make 
some effort in simplifying the search process for existing functionality,
not only because one can avoid the cost of re-developing the functionality,
but also because it will help reduce the code bulk which has to be 
maintained over time (as well as the size of executables).
Even tasks like testing and performance tuning are simpler with
less code bulk.

Packaging can help significantly in simplifying the search for existing
functionality, especially if the package hierarchy is along the lines
of functionality and responsibility.

Documentation tools may present class documentation along in
a hierarchical form which mirrors the package structure.

One may also want to use Java's idea of mapping the package 
(namespace) hierarchy onto a directory hierarchy. This may
also help to simply the search for components.

%------------------------------------------------------------------

\section{Exercises}

Decide on a sensible package hierarchy for the elements of the exercise
of chapter \ref{chapAbstraction} and insert the components into the
hierarchy. Test that your application still compiles and runs.
