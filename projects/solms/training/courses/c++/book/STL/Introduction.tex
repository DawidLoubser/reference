\section{Introduction and Overview}

The Standard Template Library (STL) is C++'s collection class library. It
supports most classical data structures like linked lists, maps, queues
and automatically resizing arrays. 

The Java 2 Collection Framework itself is largely based on the STL. Though
it is in some ways more powerful, it is in other ways less general than 
the STL.

The library consists of a collection of heavily parametrized classes and 
functions. It is built around 3 core pillars:
\begin{description}
  \item[Containers] stores the actual elements and may provide methods
                    for accessing elements.
  \item[Iterators]  are generalizations of pointers. They can be used
                    to step through a collection and to retrieve, 
                    modify or insert an element at the position they
                    are currently pointing to,
  \item[Algorithms] are general operations which may be performed on
                    containers. This includes searching, sorting, copying,
                    filling and more.
\end{description}                                        
                   
%--------------------------------------------------------------------------

\subsection{Some Core Design Decisions}

A lot of effort has been put into the design of the STL and many of these
design ideas have since been directly taken over by other collection class
libraries like the Java 2 Collection Framework.

%----------------------------------

\subsubsection{The design is centered around interfaces}

Like the Java 2 Collection Framework, the STL is designed around
interfaces and not around the implementation classes.


%----------------------------------

\subsubsection{Algorithms are defined seperately from the container classes}

The STL follows an un-usual design from an object-oriented perspective and
this design has been mirrored in the Java 2 Collection Framework in that the 
algorithms are supplied as stand-alone functions and not as instance 
methods of the container classes.

%----------------------------------

\subsubsection{Iterators are modelled as specializations of pointers}

One of the design decisions made for the STL is to define iterators as
specializations of pointers, i.e.\ iterators support dereferencing as
well as pointer arithmetic. This makes it possible that the algorithms
are equally applicable to the collection classes and to primitive
pointer-based arrays.
